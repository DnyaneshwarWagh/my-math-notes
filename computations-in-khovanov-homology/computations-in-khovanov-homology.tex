

\input{"/Users/brandonwilliams/Documents/LaTeX Includes/hwpreamble.tex"}
\input{"/Users/brandonwilliams/Documents/LaTeX Includes/extrapackages.tex"}
\input{"/Users/brandonwilliams/Documents/LaTeX Includes/extracommands.tex"}


\begin{document}


\title{\Large Computations in Khovanov Homology}
\author{\large Brandon Williams}

\maketitle


\section{Notation}


Here we quickly fix our notation. Let $\mathcal A$ denote Khovanov's TQFT. We let $V$ be the free, graded $\mathbb Z$-module generated by $v_+$ and $v_-$, which have degrees $p(v_\pm) = \pm 1$. The multiplication and comultiplication induced on $V$ from cobordisms is denoted by $m$ and $\Delta$\, where
\[ m(v_+ \otimes v_+) = v_+ \ \ \ \ \ m(v_\pm \otimes v_\mp) = v_- \ \ \ \ \ m(v_- \otimes v_-) = 0 \]
\[ \Delta(v_+) = v_+ \otimes v_- + v_- \otimes v_+ \ \ \ \ \ \Delta(v_-) = v_- \otimes v_- \]
Let $L$ be a link with $k$ crossings and $(n_+,n_-)$ positive/negative crossings. If $v \in \lcb 0,1 \rcb^k$ we let $L_v$ denote the smoothing of $L$, elements $x$ of $\mathcal A(L_v)$ are said to have homological grading $\gr(x) = c_1(v) - n_-$, where $c_1(v)$ is the number of 1's in the coordinates of $v$. Elements $x$ of $\mathcal A(L_v)$ of homogenous $p$-grading are said to have quantum grading $q(x) = p(x) + \gr(x) + n_+ - n_-$. The Khovanov chain complex is denoted by $\CKh^{i,j}$, where the first index is the homological grading and the second index is the quantum grading. The homology of this complex is denoted by $\Kh^{i,j}$. The graded Euler characteristic is defined to be the Laurent polynomial
\[ \Kh(q) = \sum_j \left( \sum_i (-1)^i \rank \Kh^{i,j} \right) q^j \]



\section{A Useful Trick}


Before making some computations we first discuss a trick that will be used repeatedly in what is to come. It basically deals with determining the quotient of free $\mathbb Z$-modules by free $\mathbb Z$-modules. For instance, we know that the free module $\mathbb Z$ modulo the free module $2\mathbb Z$ is simple $\mathbb Z/2$. We will be running into much more complicated instances of this in our computations.

Let $M$ be a submodule of $\mathbb Z^n$, which must be free. Let $v_1,\ldots,v_k$ be a basis for $M$, and let $A$ be the matrix whose columns consists of the vectors $v_i$, and add some extra zero columns to make $A$ and $n \times n$ matrix (if needed). Then we claim that $\mathbb Z^n / M$ is isomorphic to $\mathbb Z/(d_1) \oplus \cdots \mathbb Z/(d_n)$, where $d_i$ are the eigenvalues of $A$. 

As an example, suppose we want to mod $\mathbb Z^3$ by the subspace generated by the vectors $(1,0,1),(0,1,0)$. The matrix $A$ is given by
\[ A = \left( \begin{array}{ccc} 1 & 0 & 0 \\ 0 & 1 & 0 \\ 1 & 0 & 0 \end{array} \right) \]
The eigenvalues of this can easily be computed to be $2,1,0$, hence
\[ \frac{\mathbb Z^3}{\< (1,0,1), (0,1,0) \>} = \mathbb Z/(2) \oplus \mathbb Z/(1) \oplus \mathbb Z/(0) = \mathbb Z/2 \oplus \mathbb Z \]

Most of the messy computations in Khovanov homology can be reduced to solving something of this form, so this trick will be quite useful for us.



\section{The Hopf Link}


Let $L$ be the Hopf link with orientation such that both crossings are negative. For each $v \in \lcb 0,1 \rcb^2$, let $L_v$ be the resolution of $L$. Using the usual TQFT we get a chain complex
\[ 0 \longrightarrow V \otimes V \stackrel{d_1}{\longrightarrow} V \oplus V \stackrel{d_2}{\longrightarrow} V \otimes V \longrightarrow 0 \]
It is easy to see that the differentials are given by
\begin{align}
	d_1(v_1 \otimes v_2) &= (m(v_1 \otimes v_2), m(v_1 \otimes v_2)) \\
	d_2(v_1,v_2) &= \Delta(v_1) - \Delta(v_2) 
\end{align}
We want to write these maps in matrix form, so we fix an ordered basis. Let the ordered basis for the spaces in the chain complex be given by writting:
\[ V \otimes V = \mathbb Z v_+ \otimes v_+ \oplus \mathbb Z v_- \otimes v_+ \oplus \mathbb Z v_+ \otimes v_- \oplus \mathbb Z v_- \otimes v_- \]
\[ V \oplus V = \mathbb Z^2 v_+ \oplus \mathbb Z^2 v_- \]
Writing our modules this way determines an ordered basis. For example, in $V \otimes V$ we have the basis 
\begin{align*}
e_1 &= v_+ \otimes v_+ \\
e_2 &= v_- \otimes v_+ \\
e_3 &= v_+ \otimes v_- \\
e_4 &= v_- \otimes v_-
\end{align*}
whereas in $V \oplus V$ we have the basis
\begin{align*}
e_1 &= (1,0)v_+ \\
e_2 &= (0,1)v_+ \\
e_3 &= (1,0)v_- \\
e_4 &= (0,1)v_- 
\end{align*}

By plugging each vector into $d_1$ or $d_2$ we get a matrix representation of these maps. In particular we can easily compute:
\[
d_1 = \left( 
			 	 \begin{array}{cccc} 
			 	 	 1 & 0 & 0 & 0 \\
			 	 	 1 & 0 & 0 & 0 \\
			 	 	 0 & 1 & 1 & 0 \\
			 	 	 0 & 1 & 1 & 0
			 	 \end{array} 
			\right)
\ \ \ \ \ \ \ 
d_2 = \left( 
			 	 \begin{array}{cccc} 
			 	 	 0 &  0 & 0 &  0 \\
			 	 	 1 & -1 & 0 &  0 \\
			 	 	 1 & -1 & 0 &  0 \\
			 	 	 0 &  0 & 1 & -1
			 	 \end{array} 
			\right)
\]
Now we compute the image and kernels of each of these maps. Clearly the image of $d_1$ is spanned by the first two columns, which corresponds to (remember our notation for basis vectors in $V \oplus V$ above, otherwise things will be confusing):
\[ \image d_1 = \mathbb Z(1,1)v_+ \oplus \mathbb Z(1,1)v_- \]
For the kernel, it is easy to see that if $(a,b,c,d)$ is a column vector in the kernel of the matrix of $d_1$, then $a=0,b=-c$, so the kernel is spanned by the column vectors $(0,1,-1,0)$ and $(0,0,0,1)$. The former vector spans the subspace $\mathbb Z (v_+ \otimes v_- - v_- \otimes v_+)$ and the latter vector spaces the subspace $\mathbb Z v_- \otimes v_-$, so
\[ \ker d_1 = \mathbb Z (v_+ \otimes v_- - v_- \otimes v_+) \oplus \mathbb Z v_- \otimes v_- \]
Next we look at $d_2$. Clearly the image of $d_2$ is spanned by the first and third columns, which corresponds to
\[ \image d_2 = \mathbb Z(v_+ \otimes v_- + v_- \otimes v_+) \oplus \mathbb Z v_- \otimes v_- \]
For the kernel, it is easy to see that if $(a,b,c,d)$ is in the kernel of the matrix of $d_2$, then $a = b$ and $c=d$, so the kernel is spanned by the vectors $(1,1,0,0)$ and $(0,0,1,1)$. The former vector spans the subspace $\mathbb Z (1,1)v_+$ and the latter vector spans the subspace $\mathbb Z (1,1)v_-$, so
\[ \ker d_2 = \mathbb Z(1,1)v_+ \oplus \mathbb Z(1,1)v_- \]
Now we compute the homologies:
\begin{align*}
\Kh^{-2} &= \ker d_1 = \mathbb Z (v_+ \otimes v_- - v_- \otimes v_+) \oplus \mathbb Z v_- \otimes v_- \\
\Kh^{-1} &= \frac{\ker d_2}{\image d_1} = \frac{\mathbb Z(1,1)v_+ \oplus \mathbb Z(1,1)v_-}{\mathbb Z(1,1)v_+ \oplus \mathbb Z(1,1)v_-} = 0 \\
\Kh^0    &= \frac{V \otimes V}{\image d_2} = \frac{\mathbb Z v_+ \otimes v_+ \oplus \mathbb Z v_- \otimes v_+ \oplus \mathbb Z v_+ \otimes v_- \oplus \mathbb Z v_- \otimes v_-}{\mathbb Z(v_+ \otimes v_- + v_- \otimes v_+) \oplus \mathbb Z v_- \otimes v_-} \\
         &= \mathbb Z v_+ \otimes v_+ \oplus \frac{\mathbb Z v_- \otimes v_+ \oplus \mathbb Z v_+ \otimes v_-}{\mathbb Z(v_+ \otimes v_- + v_- \otimes v_+)} 
\end{align*}
The quotient in the last line is simply isomorphic to $\mathbb Z$ since we are essentially taking the quotient of $\mathbb Z \oplus \mathbb Z$ by the diagonal, which is just $\mathbb Z$ (because there is an obvious surjection $\mathbb Z \oplus \mathbb Z \rightarrow \mathbb Z$ with kernel equal to the diagonal). This finishes the computation of the Khovanov homology of the link, but now we want to determine the quantum gradings. It is easy to see that (where $r$ stands for the homological grading and $q$ stands for the quantum grading)
\[
\begin{tabular}{|r|c|c|c|}
\hline
$\Kh^{r,q}$ & $-2$ & $-1$ & $0$ \\
\hline
$0$ & & & $\mathbb Z$  \\
\hline
$-1$ & & &   \\
\hline
$-2$ & & & $\mathbb Z$  \\
\hline
$-3$ & & &   \\
\hline
$-4$ & $\mathbb Z$ &    &   \\
\hline
$-5$ & &    &   \\
\hline
$-6$ & $\mathbb Z$ &    &   \\
\hline
\end{tabular}
\]
The graded Euler characteristic of this homology is
\[ \Kh(L)(q) = q^{-6} + q^{-4} + q^{-2} + 1 \]
This should be $(q+q^{-1})$ times the Jone's polynomial of $L$, so factoring out a $(q+q^{-1})$ from the above we get
\begin{align*}
	q^{-6} + q^{-4} + q^{-2} + 1 &= q^{-4}(q^{-2} + 1) + q^{-2} + 1 \\
	                             &= q^{-5}(q^{-1} + q) + q^{-1}(q^{-1} + q) \\
	                             &= (q+q^{-1})(q^{-5} + q^{-1})
\end{align*}
and $q^{-5} + q^{-1}$ is indeed the Jones polynomial (up to some substitution of variable).



\section{The Left-Handed Trefoil}

Let $L$ be the left handed trefoil (i.e. a trefoil with orientation such that all crossings are negative). Then our chain complex is
\[ 0 \longrightarrow V \otimes V \otimes V \stackrel{d_1}{\longrightarrow} (V \otimes V) \oplus (V \otimes V) \oplus (V \otimes V) \stackrel{d_2}{\longrightarrow} V \oplus V \oplus V \stackrel{d_3}{\longrightarrow} V \otimes V \longrightarrow 0 \]
with differentials given by
\begin{align*}
d_1(v_1 \otimes v_2 \otimes v_3) &= (m(v_1 \otimes v_2) \otimes v_3, v_1 \otimes m(v_2 \otimes v_3), v_2 \otimes m(v_1 \otimes v_3)) \\
d_2(v_1 \otimes v_2,v_3 \otimes v_4,v_5 \otimes v_6) &= (m(v_3 \otimes v_4)-m(v_1 \otimes v_2), m(v_5 \otimes v_6)-m(v_1 \otimes v_2), m(v_5 \otimes v_6)-m(v_3 \otimes v_4)) \\
d_3(v_1,v_2,v_3) &= \Delta(v_1) - \Delta(v_2) + \Delta(v_3)
\end{align*}
We fix an ordered basis for the chain vector spaces by writting
\begin{align*}
\begin{array}{rcl}
V \otimes V \otimes V &=& \mathbb Z v_+ \otimes v_+ \otimes v_+ \oplus \mathbb Z v_- \otimes v_+ \otimes v_+ \oplus \mathbb Z v_+ \otimes v_- \otimes v_+ \oplus \\
										  & & \mathbb Z v_+ \otimes v_+ \otimes v_- \oplus \mathbb Z v_- \otimes v_- \otimes v_+ \oplus \mathbb Z v_- \otimes v_+ \otimes v_- \oplus \\
										  & & \mathbb Z v_+ \otimes v_- \otimes v_- \oplus \mathbb Z v_- \otimes v_- \otimes v_- \\
(V \otimes V) \oplus (V \otimes V) \oplus (V \otimes V) &=& \mathbb Z^3 v_+ \otimes v_+ \oplus \mathbb Z^3 v_- \otimes v_+ \oplus \mathbb Z^3 v_+ \otimes v_- \oplus \mathbb Z^3 v_- \otimes v_- \\
V \oplus V \oplus V &=& \mathbb Z^3 v_+ \oplus \mathbb Z^3 v_- \\
V \otimes V &=& \mathbb Z v_+ \otimes v_+ \oplus \mathbb Z v_- \otimes v_+ \oplus \mathbb Z v_+ \otimes v_- \oplus \mathbb Z v_- \otimes v_-
\end{array}
\end{align*}
We now take the matrix representation of $d_1,d_2,d_3$ with respect to these bases.
\[ d_1 = \left(
					 \begin{array}{cccccccc}
					   1 & 0 & 0 & 0 & 0 & 0 & 0 & 0 \\
					   1 & 0 & 0 & 0 & 0 & 0 & 0 & 0 \\
					   1 & 0 & 0 & 0 & 0 & 0 & 0 & 0 \\
					   0 & 1 & 1 & 0 & 0 & 0 & 0 & 0 \\
					   0 & 1 & 0 & 0 & 0 & 0 & 0 & 0 \\
					   0 & 0 & 1 & 0 & 0 & 0 & 0 & 0 \\
					   0 & 0 & 0 & 1 & 0 & 0 & 0 & 0 \\
					   0 & 0 & 1 & 1 & 0 & 0 & 0 & 0 \\
					   0 & 1 & 0 & 1 & 0 & 0 & 0 & 0 \\
					   0 & 0 & 0 & 0 & 0 & 1 & 1 & 0 \\
					   0 & 0 & 0 & 0 & 1 & 1 & 0 & 0 \\
					   0 & 0 & 0 & 0 & 1 & 0 & 1 & 0 
					 \end{array}
				 \right)
\ \ \ \ \ \ \ \ 
d_2 = \left(
        \begin{array}{cccccccccccc}
        	-1 &  1 & 0 &  0 &  0 & 0 &  0 &  0 & 0 & 0 & 0 & 0 \\
        	-1 &  0 & 1 &  0 &  0 & 0 &  0 &  0 & 0 & 0 & 0 & 0 \\
        	 0 & -1 & 1 &  0 &  0 & 0 &  0 &  0 & 0 & 0 & 0 & 0 \\
        	 0 &  0 & 0 & -1 &  1 & 0 & -1 &  1 & 0 & 0 & 0 & 0 \\
        	 0 &  0 & 0 & -1 &  0 & 1 & -1 &  0 & 1 & 0 & 0 & 0 \\
        	 0 &  0 & 0 &  0 & -1 & 1 &  0 & -1 & 1 & 0 & 0 & 0 
        \end{array}
      \right)
\]
\[
d_3 = \left(
        \begin{array}{cccccc}
          0 &  0 & 0 & 0 &  0 & 0 \\
          1 & -1 & 1 & 0 &  0 & 0 \\
          1 & -1 & 1 & 0 &  0 & 0 \\
          0 &  0 & 0 & 1 & -1 & 1 
        \end{array}
      \right)
\]
We now want to compute the image and kernels of these matrices. It is easy to see that the first 7 columns of $d_1$ are linearly independent, and the subspaces they span are easy to determine, so the image is (again remember our convention for basis vectors from above, otherwise this notation we seem weird)
\begin{align}
\label{image d1}
\begin{array}{rcl}
\image d_1 &=& \mathbb Z \left( (1,0,0)v_+ \otimes v_+ + (0,1,0)v_+ \otimes v_+ + (0,0,1)v_+ \otimes v_+ \right) \oplus \\
           & & \mathbb Z \left( (1,0,0)v_- \otimes v_+ + (0,1,0)v_- \otimes v_+ + (0,0,1)v_+ \otimes v_- \right) \oplus \\
           & & \mathbb Z \left( (1,0,0)v_- \otimes v_+ + (0,1,0)v_+ \otimes v_- + (0,0,1)v_- \otimes v_+ \right) \oplus \\
           & & \mathbb Z \left( (1,0,0)v_+ \otimes v_- + (0,1,0)v_+ \otimes v_- + (0,0,1)v_+ \otimes v_- \right) \oplus \\
           & & \mathbb Z \left( (0,1,0)v_- \otimes v_- + (0,0,1)v_- \otimes v_- \right) \oplus \\
           & & \mathbb Z \left( (1,0,0)v_- \otimes v_- + (0,1,0)v_- \otimes v_- \right) \oplus \\
           & & \mathbb Z \left( (1,0,0)v_- \otimes v_- + (0,0,1)v_- \otimes v_- \right) \oplus 
\end{array}
\end{align}
The kernel of $d_2$ is easy to compute, indeed only the vectors of the form $(0,0,0,0,0,0,0,h)$ are in the kernel, so
\[ \ker d_1 = \mathbb Z v_- \otimes v_- \otimes v_- \]
Next we compute the image of $d_2$. It is easy to see that only the 1st, 2nd, 4th and 5th columns are linearly independent, and they span the subspace
\begin{align*}
\image d_2 =& \mathbb Z \left( (1,0,0) v_+ + (0,1,0) v_+ \right) \oplus \mathbb Z \left( (1,0,0) v_+  - (0,0,1) v_+ \right) \\
            & \mathbb Z \left( (1,0,0) v_- + (0,1,0) v_- \right) \oplus \mathbb Z \left( (1,0,0) v_-  - (0,0,1) v_- \right) 
\end{align*}
Computing the kernel is a bit messy, but we can use a number of methods (row reducing, Mathematica, etc.), and so we just state the spanning vectors of the kernel:
\[
\left( \begin{array}{r} 1\\1\\1\\0\\0\\0\\0\\0\\0\\0\\0\\0 \end{array} \right), 
\left( \begin{array}{r} 0\\0\\0\\1\\1\\1\\0\\0\\0\\0\\0\\0 \end{array} \right), 
\left( \begin{array}{r} 0\\0\\0\\-1\\0\\0\\1\\0\\0\\0\\0\\0 \end{array} \right), 
\left( \begin{array}{r} 0\\0\\0\\0\\-1\\0\\0\\1\\0\\0\\0\\0 \end{array} \right), 
\left( \begin{array}{r} 0\\0\\0\\1\\1\\0\\0\\0\\1\\0\\0\\0 \end{array} \right), 
\left( \begin{array}{r} 0\\0\\0\\0\\0\\0\\0\\0\\0\\1\\0\\0 \end{array} \right), 
\left( \begin{array}{r} 0\\0\\0\\0\\0\\0\\0\\0\\0\\0\\1\\0 \end{array} \right), 
\left( \begin{array}{r} 0\\0\\0\\0\\0\\0\\0\\0\\0\\0\\0\\1 \end{array} \right)
\]
These vectors span the following space
\begin{align}
\label{ker d2}
\begin{array}{rcl}
\ker d_2 &=& \mathbb Z \left(  (1,0,0) v_+ \otimes v_+ + (0,1,0) v_+ \otimes v_+ + (0,0,1) v_+ \otimes v_+ \right) \oplus \\
         & & \mathbb Z \left(  (1,0,0) v_- \otimes v_+ + (0,1,0) v_- \otimes v_+ + (0,0,1) v_- \otimes v_+ \right) \oplus \\
         & & \mathbb Z \left( -(1,0,0) v_- \otimes v_+ + (1,0,0) v_+ \otimes v_- \right) \oplus \\
         & & \mathbb Z \left( -(0,1,0) v_- \otimes v_+ + (0,1,0) v_+ \otimes v_- \right) \oplus \\
         & & \mathbb Z \left(  (1,0,0) v_- \otimes v_+ + (0,1,0) v_- \otimes v_+ + (0,0,1) v_+ \otimes v_- \right) \oplus \\
         & & \mathbb Z (1,0,0) v_- \otimes v_- \oplus \\
         & & \mathbb Z (0,1,0) v_- \otimes v_- \oplus \\
         & & \mathbb Z (0,0,1) v_- \otimes v_- 
\end{array}
\end{align}
Finally we need to find the image and kernel of $d_3$, which thankfully is easier than the previous matrices. It is clear from inspection that only the 1st and 4th columns of $d_3$ are linear independent, and so they span the image
\[ \image d_3 = \mathbb Z (v_- \otimes v_+ + v_+ \otimes v_-) \oplus \mathbb Z v_- \otimes v_- \]
We can compute the kernel to be spanned by the vectors
\[
\left( \begin{array}{r} 1\\1\\0\\0\\0\\0 \end{array} \right), 
\left( \begin{array}{r} -1\\0\\1\\0\\0\\0 \end{array} \right), 
\left( \begin{array}{r} 0\\0\\0\\1\\1\\0 \end{array} \right), 
\left( \begin{array}{r} 0\\0\\0\\-1\\0\\1 \end{array} \right)
\]
These vector span the space
\begin{align*}
\ker d_3 =& \mathbb Z \left( (1,0,0) v_+ + (0,1,0) v_+ \right) \oplus \mathbb Z \left( -(1,0,0) v_+ + (0,0,1) v_+ \right) \\
          & \mathbb Z \left( (1,0,0) v_- + (0,1,0) v_- \right) \oplus \mathbb Z \left( -(1,0,0) v_- + (0,0,1) v_- \right)
\end{align*}
We now take quotients to compute the homology. The first homology group (meaning the one with the lowest homological degree) is easy enough to compute:
\[ H^{-3} = \ker d_1 = \mathbb Z v_- \otimes v_- \otimes v_- \]
The next group is more complicated to compute, mainly because a look back at $\ker d_2$ and $\image d_1$ shows that they are quite complicated. However, we see that there is a $\mathbb Z^3 v_+ \otimes v_+$ term in $\ker d_2$ and $\image d_1$, so those terms cancel when taking the quotient. Also, it is clear that the last three summands of $\ker d_2$ in \eqref{ker d2} make up all of $\mathbb Z^3 v_- \otimes v_-$ in $(V \otimes V)^3$, whereas the last three summands of $\image d_1$ in \eqref{image d1} span the subspace of $\mathbb Z^3$ generated by $(0,1,1),(1,1,0)$ and $(1,0,1)$. By our useful trick mentioned earlier, we have that the quotient of $\mathbb Z^3$ by the subspace generated by $(0,1,1),(1,1,0)$ and $(1,0,1)$ is isomorphic to $\mathbb Z/d_1 \oplus \mathbb Z/d_2 \oplus \mathbb Z/d_3$, where $d_1,d_2,d_3$ are the eigenvalues of the matrix formed by those vectors. So we have
\[ \frac{\mathbb Z^3}{\< (0,1,1),(1,1,0),(1,0,1) \>} = \mathbb Z/2 \]
Now we have to worry about the stuff left over in $\ker d_2$ and $\image d_1$. This boils down to finding the the following quotient
\[ \frac{ \< \left( \begin{array}{r} 1\\1\\0\\0\\0\\1 \end{array} \right),
             \left( \begin{array}{r} 1\\1\\1\\0\\0\\0 \end{array} \right),
             \left( \begin{array}{r} -1\\0\\0\\1\\0\\0 \end{array} \right),
             \left( \begin{array}{r} 0\\-1\\0\\0\\1\\0 \end{array} \right)
           \>}{ \< 
             \left( \begin{array}{r} 1\\1\\0\\0\\0\\1 \end{array} \right), 
             \left( \begin{array}{r} 1\\0\\1\\0\\1\\0 \end{array} \right), 
             \left( \begin{array}{r} 0\\0\\0\\1\\1\\1 \end{array} \right) 
\> } \]
We can't apply our trick in this form though. To make this nicer we can just change the basis: let $v_1,v_2,v_3,v_4$ be the vectors in the numerator (in order), then the bottom vectors can be expressed as $v_1,v_2+v_4,v_1+v_3+v_4$ respectively. Therefore determining the above quotient has become equivalent to find
\[ \frac{ \< \left( \begin{array}{r} 1\\0\\0\\0\\0\\0 \end{array} \right),
             \left( \begin{array}{r} 0\\1\\0\\0\\0\\0 \end{array} \right),
             \left( \begin{array}{r} 0\\0\\1\\0\\0\\0 \end{array} \right),
             \left( \begin{array}{r} 0\\0\\0\\1\\0\\0 \end{array} \right)
           \>}{ \< 
             \left( \begin{array}{r} 1\\0\\0\\0\\0\\0 \end{array} \right), 
             \left( \begin{array}{r} 0\\1\\0\\1\\0\\0 \end{array} \right), 
             \left( \begin{array}{r} 1\\0\\1\\1\\0\\0 \end{array} \right) 
\> } \]
The bottom two rows on these vectors is zero, so this is equivalent to determining
\[ \frac{ \< \left( \begin{array}{r} 1\\0\\0\\0 \end{array} \right),
             \left( \begin{array}{r} 0\\1\\0\\0 \end{array} \right),
             \left( \begin{array}{r} 0\\0\\1\\0 \end{array} \right),
             \left( \begin{array}{r} 0\\0\\0\\1 \end{array} \right)
           \>}{ \< 
             \left( \begin{array}{r} 1\\0\\0\\0 \end{array} \right), 
             \left( \begin{array}{r} 0\\1\\0\\1 \end{array} \right), 
             \left( \begin{array}{r} 1\\0\\1\\1 \end{array} \right) 
\> }
= \frac{\mathbb Z^4}{ \< (1,0,0,0),(0,1,0,1),(1,0,1,1) \> }
\]
Applying our trick shows that this is isomorphic to $\mathbb Z$. Compiling all of this information we have determined the next homology group
\[ H^{-2} = \mathbb Z \oplus \mathbb Z/2 \]
The next group is easier to compute. An easy inspection shows that $\ker d_3 = \image d_2$, so we have
\[ H^{-1} = \frac{\ker d_3}{\image d_2} = 0 \]
The last group is also pretty easy compute as it is reminiscent of the last homology group in the Hopf link.
\begin{align*}
H^0    =& \frac{V \otimes V}{\image d_3} = \frac{\mathbb Z v_+ \otimes v_+ \oplus \mathbb Z v_- \otimes v_+ \oplus \mathbb Z v_+ \otimes v_- \oplus \mathbb Z v_- \otimes v_-}{\mathbb Z (v_- \otimes v_+ + v_+ \otimes v_-) \oplus \mathbb Z v_- \otimes v_-} \\
       =& \mathbb Z v_+ \otimes v_+ \oplus \frac{\mathbb Z v_- \otimes v_+ \oplus \mathbb Z v_+ \otimes v_-}{\mathbb Z (v_- \otimes v_+ + v_+ \otimes v_-)}
\end{align*}
The last quotient is isomorphic to $\mathbb Z$ since it is $\mathbb Z \oplus \mathbb Z$ modulo the diagonal subgroup. Computing the quantum gradings we get the following table:
\[
\begin{tabular}{|r|c|c|c|c|}
\hline
$\Kh^{r,q}$ & $-3$ & $-2$ & $-1$ & $0$ \\
\hline
$-1$ & & & & $\mathbb Z$ \\
\hline
$-2$ & & & &  \\
\hline
$-3$ & & & & $\mathbb Z$ \\
\hline
$-4$ & & & &  \\
\hline
$-5$ & & $\mathbb Z$ & &  \\
\hline
$-6$ & & & &  \\
\hline
$-7$ & & $\mathbb Z/2$ & &  \\
\hline
$-8$ & & & &  \\
\hline
$-9$ & $\mathbb Z$ & & &  \\
\hline
\end{tabular}
\]



\end{document}





