

\input{"/Users/brandonwilliams/Documents/LaTeX Includes/notespreamble.tex"}
\input{"/Users/brandonwilliams/Documents/LaTeX Includes/extrapackages.tex"}
\input{"/Users/brandonwilliams/Documents/LaTeX Includes/extracommands.tex"}


\begin{document}


\title{\Large Gauge Theory}
\author{\large Brandon Williams}

\maketitle


\tableofcontents


\newpage



\section{Introduction}


These notes cover the theory of vector and principal bundles from a pure mathematics viewpoint, with which the author hopes to later understand some physics and gauge theory. We will use the summation convention through these notes, and we always work in the smooth category.








\section{Some Linear Algebra}
\label{Some Linear Algebra}



For an $n$-dimensional vector space $V$, the space of \textbf{tensors} of type $(r,s)$ on $V$ will be denoted by
\[ V_{r,s} := \underbrace{V \otimes \cdots \otimes V}_{r \text{ times}} \otimes \underbrace{V^* \otimes \ldots \otimes V^*}_{s \text{ times}} \]
The \textbf{tensor algebra} $T(V)$ over $V$ is the direct sum of all of the spaces, i.e.
\[ T(V) := \bigoplus_{r,s \geq 0} V_{r,s} \]
Its a non-commutative, associated, graded algebra, with multiplication given by $\otimes$. Let $C(V)$ denote the sub-algebra of pure covariant tensors, i.e.
\[ C(V) := \bigoplus_{r \geq 0} V_{r,0} \]
Let $I(V)$ be the two-sided ideal in $C(V)$ generated by all elements of the form $u \otimes u$. This is a graded ideal, where we set $I_k(V) = I(V) \cap V_{k,0}$. Then we define the $k$-th \textbf{exterior power} of $V$ to be the vector space $\Lambda^k(V) = V_{k,0} / I_k(V)$, and define the \textbf{exterior algebra} of $V$ to be the algebra
\[ \Lambda(V) := C(V) / I(V) \]
This inherits a multiplication structure by taking the equivalence class of the tensor product of representatives of two equivalence classes. We denote this multiplication by $\wedge$, hence the equivalence class of $v_1 \otimes \cdots \otimes v_r$ is denoted by $v_1 \wedge \cdots \wedge v_r$. 



A \textbf{pairing} of vector spaces $V$ and $W$ is a bilinear map $( \ , \ ) : V \times W \rightarrow \mathbb R$. A pairing is said to be \textbf{non-degenerate} if for each non-zero $v \in V$ there is a $w \in W$ such that $(v,w) \neq 0$. This is equivalent to saying that the association $v \mapsto (v,-)$ defines an isomorphism $V \cong W^*$. Suppose $(\ ,\ )$ is a non-degenerate pairing on $V$ and $W$ so that $V$ is isomorphic to $W^*$, and suppose these vector spaces are of dimension $n$. The process of turning vectors in $V$ into covectors in $W$ is called lowering an index (and the inverse is called raising an index). Let us look at this process in a basis.

Fix a basis $(e_i)$ and $(f_i)$ for $V$ and $W$ respectively, and let $(e^i)$ and $(f^i)$ denote the dual bases. This determines an $n$ by $n$ matrix $g_{ij} = (e_i,f_j)$, which is non-singular since the pairing is non-degenerate. For $v = v^i e_i$ in $V$ let $\lambda = \lambda_i f^i$ denote the covector corresponding to $v$, i.e. $\lambda = (v,-)$. Then 
\begin{equation}
\label{pairing - lowering index}
\lambda_j = \lambda(f_j) = \left( v^i e_i, f_j \right) = g_{ij} v^i
\end{equation}
So, $\lambda$ in the basis $(f^i)$ is equal to $v$ in the basis $(e_i)$ multiplied by the transpose of the matrix $(g_{ij})$. On the other hand, if we knew the covector $\lambda$ and wanted to find the corresponding vector $v$ in $V$, we would need to multiply by the inverse of the transpose of $(g_{ij})$. However, the inverse of the transpose is the transpose of the inverse, so we have that the components of $v$ in terms of the components of $\lambda$ are given by
\begin{equation}
\label{pairing - raising index}
v^j = g^{ij} \lambda_i
\end{equation}
To make sure this is correct we see that
\[ v^j = g^{ij} \lambda_i = g^{ij} g_{ki} v^k = \delta_k^j v^k = v^j \]


Non-degenerate pairings most often come in the form of inner products. Let $\<\ ,\ \>$ be an inner product on $V$, that is, $\<\ ,\ \>$ is a symmetric, bilinear, non-degenerate pairing on $V \times V$ (note that we are allowing indefinite signature here). We have a canonical isomorphism $V \cong V^*$, and the process of converting vectors (covectors) to covectors (vectors) is again called lowering (raising) an index. This time, however, the matrix $g_{ij} = \< e_i,e_j \>$ in a basis is symmetric, so the transpose of $(g_{ij})$ is equal to itself, and so \ref{pairing - lowering index} and \ref{pairing - raising index} can be written as
\begin{equation}
\label{lowering index}
\lambda_i = g_{ij} v^j
\end{equation}
\begin{equation}
\label{raising index}
v^i = g^{ij} \lambda_j
\end{equation}
When written like this we see that lowering (or raising) an index means to multiply by the inner product matrix (or its inverse) and contract a pair of indices. More generally, given any tensor $F \in V_{r,s}$ we can use the metric to lower or raise any index. For example, suppose $F \in V \otimes V^* \otimes V \otimes V^*$ has components $F^{i\ k}_{\ j\ l}$ and we wanted to lower the first index to get the corresponding tensor in $V^* \otimes V^* \otimes V \otimes V^*$. This tensor would have components $F^{\ \ k}_{ij\ l}$ defined by
\[ F^{\ \ k}_{ij\ \ l} := g_{im} F^{m\ k}_{\ \ j\ \ l} \]
Now suppose we wanted to raise the last index to get the corresponding tensor in $V^* \otimes V^* \otimes V \otimes V$. This tensor has component $F_{ij}^{\ \ kl}$ defined by
\[ F_{ij}^{\ \ kl} := g^{ln} F^{\ \ k}_{ij\ \ n} = g^{ln} g_{im} F^{m\ k}_{\ \ j\ \ n} \]



The inner product $\<\ ,\ \>$ can be extended to an inner product on all the vector spaces we construct from $V$, such as $V^*$, $V \otimes V$, and more generally $V_{r,s}$. For example, if $\lambda,\eta \in V^*$ are covectors, then we can define $\<\lambda,\eta\>$ to simply be $\<v,w\>$, where $v,w$ are the corresponding vectors of $\lambda,\eta$ from raising an index. In components, if $\lambda = \lambda_i e^i$ and $\eta = \eta_i e^i$, then $v = g^{ij} \lambda_j e_i$ and $w = g^{ij} \eta_j e_i$ are the corresponding vectors, and so
\[ \< \lambda,\eta \> := \< v,w \> = \< g^{ij} \lambda_j e_i, g^{kl} \eta_l e_k \> = g^{ij} g^{kl} g_{ik} \lambda_j \eta_l = g^{ij} \delta_i^l \lambda_j \eta_l = g^{ij} \lambda_i \eta_j \]
On the other hand, for $V \otimes V$ we can first define $\<\ ,\ \>$ on the simple tensors, $v \otimes w$, and extend linearly. So we define $\< v_1 \otimes w_1, v_2 \otimes w_2 \> = \<v_1,w_1\>\<v_2,w_2\>$, then if $F = F^{ij} e_i \otimes e_j$ and $G = G^{ij} e_i \otimes e_j$, then 
\[ \< F,G \> := \< F^{ij} e_i \otimes e_j, G^{kl} e_k \otimes e_l \> = F^{ij} G^{kl} g_{ik} g_{jl} \]
Putting these two examples together we can more generally say that if $F,G \in V_{r,s}$ have the following expressions
\[ F = F_{i_1 \cdots i_s}^{j_1 \cdots j_r} e_{j_1} \otimes \cdots \otimes e_{j_r} \otimes e^{i_1} \otimes \cdots \otimes e^{i_s} \]
\[ G = G_{i_1 \cdots i_s}^{j_1 \cdots j_r} e_{j_1} \otimes \cdots \otimes e_{j_r} \otimes e^{i_1} \otimes \cdots \otimes e^{i_s} \]
then
\begin{align*}
	\< F,G \> &= \< F_{i_1 \cdots i_s}^{j_1 \cdots j_r} e_{j_1} \otimes \cdots \otimes e_{j_r} \otimes e^{i_1} \otimes \cdots \otimes e^{i_s}, G_{k_1 \cdots k_s}^{l_1 \cdots l_r} e_{l_1} \otimes \cdots \otimes e_{l_r} \otimes e^{k_1} \otimes \cdots \otimes e^{k_s} \> \\
	          &:= F_{i_1 \cdots i_s}^{j_1 \cdots j_r} G_{k_1 \cdots k_s}^{l_1 \cdots l_r} g_{j_1l_1} \cdots g_{j_rl_r} g^{i_1k_1} \cdots g^{i_sk_s}
\end{align*}





We can easily compute $\dim \Lambda^p V = \binom{p}{n} = \binom{p}{n-p} = \dim \Lambda^{n-p} V$, so these vector spaces are isomorphic. There is a particularly nice isomorphism, called the Hodge star. We will give 3 equivalent definitions of the star operator, as each has its own advantages.

\begin{enumerate}

	\item The first is the most natural way of defining the star operator. Consider the pairing $(\ ,\ ) : \Lambda^p V \times \Lambda^{n-p} V \rightarrow \Lambda^n V \cong \mathbb R$ given by $(\omega,\eta) = \omega \wedge \eta$. This pairing is non-degenerate, so we get a canonical isomorphism $\Lambda^p V \rightarrow \left( \Lambda^{n-p} V \right)^*$. Since the inner product $\<\ ,\ \>$ on $V$ induces an inner product on $\Lambda^{n-p} V$, we can raise indices to get an isomorphism $\left( \Lambda^{n-p} V \right)^* \cong \Lambda^{n-p} V$, and composing these two isomorphisms gives an isomorphism, which is called the \textbf{Hodge star}, $\star : \Lambda^p V \rightarrow \Lambda^{n-p} V$.

	\item Let $(e_1,\ldots,e_n)$ be a positively oriented basis for $V$, and let $\omega = e_1 \wedge \cdots \wedge e_n$, which is a generator of the 1-dimensional space $\Lambda^n V$. Fix a tensor $\mu \in \Lambda^p V$ and define the linear map $F : \Lambda^{n-p} V \rightarrow \Lambda^n V$ by $F(\lambda) = \mu \wedge \lambda$. This map must be of the form $F(\lambda) = f(\lambda) \omega$ for some functional $f : \Lambda^{n-p} V \rightarrow \mathbb R$. Functionals on an inner product space take a special form. This follows from the the Reisz representation theorem, but we state and prove a baby version of this theorem.

\begin{prop}
\label{functionals on an inner product space}
Let $V$ be an $n$-dimensional vector space with inner product $\< , \>$. Then for any functional $f : V \rightarrow \mathbb R$ there is a vector $w \in V$ such that
\[ f(v) = \< v,w \> \]
for all $v \in V$.
\end{prop}
\begin{proof}
Let $(e_1,\ldots,e_n)$ be an orthonormal basis, that is, $\< e_i,e_j \> = \delta_{ij}$. Define $w = \sum f(e_i) e_i$. Then for any $v = \sum v_i e_i$ in $V$ we have
\[ \< v,w \> = \< \sum v_i e_i, \sum f(e_j) e_j \> = \sum v^i f(e_j) \< e_i,e_j \> = \sum v_i f(e_j) \delta_{ij} = \sum v_i f(e_i) = f(v) \]
\end{proof}

So, by \cref{functionals on an inner product space} we must have that there is some $\tilde{\mu} \in \Lambda^{n-p} V$ such that
\[ \mu \wedge \lambda = f(\lambda) = \< \lambda, \tilde{\mu} \> \]
where $\< , \>$ is the induced inner product on $\Lambda^{n-p} V$. The tensor $\tilde{\mu}$ is called the \textbf{Hodge star} of $\mu$, and is denoted by $\star \mu$. 

We show that $\star : \Lambda^p V \rightarrow \Lambda^{n-p} V$ is well-defined, that is, it does not depend on the basis $(e_i)$. Let $(e_i')$ be another positively oriented basis for $V$, and let $\omega' = e_1' \wedge \cdots \wedge e_n'$. The $\omega$ and $\omega'$ are related by
\[ \omega' = (\det T) \omega \]
where $T : V \rightarrow V$ is the linear map such that $Te_i = e_i'$. Since $(e_i)$ and $(e_i')$ are both positively oriented and orthonormal we have $\det T = 1$, hence $\omega = \omega'$. With $F$ defined as before we see that there must be a linear map $f' : \Lambda^{n-p} V \rightarrow \mathbb R$ such that $F(\lambda) = f'(\lambda) \omega' = f'(\lambda) \omega$, hence $f' = f$. Therefore the $\mu'$ constructed from \cref{functionals on an inner product space} for $f'$ will be equal to the $\tilde{\mu}$ constructed for $f$, hence $\star \mu = \tilde{\mu} = \mu'$ is well-defined.

	\item For a positively oriented, orthonormal basis $(e_i)$ define
	\[ \star (e_1 \wedge \cdots \wedge e_p) = e_{p+1} \wedge \cdots \wedge e_n \]
	This property uniquely determines $\star$ as a linear map $\Lambda^p V \rightarrow \Lambda^{n-p} V$. 
	

\end{enumerate}


Note that the $\star$ operator depends on the inner product $\<\ ,\ \>$, but we do not include this in the notation.



%%%%%
\comment{
We can easily compute $\dim \Lambda^k V = \binom{n}{k} = \binom{n}{n-k} = \dim \Lambda^{n-k} V$, so these vector spaces are isomorphic. There is a particularly nice isomorphism, called the \textbf{Hodge star}, $\star : \Lambda^k V \rightarrow \Lambda^{n-k} V$. For a positively oriented, orthonormal basis $e_1,\ldots,e_n$ of $V$ define
\[ \star (e_1 \wedge \cdots \wedge e_k) = e_{k+1} \wedge \cdots \wedge e_n \]
If $\omega \in \Lambda^k V$ has components $\omega^{i_1 \cdots i_k}$ in some basis, then $\star \omega \in \Lambda^{n-k} V$ has components
\[ (\star \omega)_{i_1 \cdots i_{n-k}} =  \]
\unfinished
}
%%%%%%














There is a certain kind of equivalence between the concept of exterior powers and alternating multilinear maps. We formulate this now. An $r$-multilinear map $f : V \times \cdots \times V \rightarrow W$ is called \textbf{alternating} if
\[ f(v_{\sigma(1)}, \ldots, v_{\sigma(r)}) = (-1)^\sigma f(v_1,\ldots,v_r) \]
for all permutations $\sigma \in S_r$, where $(-1)^\sigma$ denotes the sign of $\sigma$. When $W = \mathbb R$, the vector space of all multilinear maps will be denoted by $M_r(V)$ and the subspace of all alternating will be denoted by $\Alt_r(V)$. The most important property of exterior powers is that it solves a certain universal problem. First recall that tensor products have a universal property, and because of this property there is a natural isomorphism
\begin{equation}
\label{univeral property tensor product isomorphism}
(V_{r,0})^* \cong M_r(V)
\end{equation}
There is a natural pairing $(\ ,\ ) : V_{r,0} \times (V^*)_{r,0} \rightarrow \mathbb R$. For simple tensors of the form
\[ u  = u_1  \otimes \cdots \otimes u_r \in V \otimes \cdots \otimes V  \]
\[ v^* = v_1^* \otimes \cdots \otimes v_r^* \in V^* \otimes \cdots \otimes V^*  \]
then
\[ (u,v) = v_1^*(u_1) \cdots v_r^*(u_r) \]
It is easy to see that this gives a bilinear map and is a non-singular pairing, so if we add in the natural isomorphism from \eqref{univeral property tensor product isomorphism} we get the following three canonical isomorphisms
\[ (V^*)_{r,0} \cong (V_{r,0})^* \cong M_r(V) \]
This allows us to induce an algebra structure on $M_{r,s}(V)$ from the algebra structure of $(V^*)_{r,s}$. 

We can formulate a universal property for alternating maps, and induce an algebra structure on the space of alternating maps.
\begin{prop}
\label{universal property of alternating maps}
Let $\pi : V \times \cdots \times V \rightarrow \Lambda^k(V)$ denote the natural map $(v_1,\ldots,v_k) \mapsto v_1 \wedge \cdots \wedge v_k$. For each multilinear, alternating map $f : V \times \cdots \times V \rightarrow W$ there is a unique linear map $\tilde{f} : \Lambda^k(V) \rightarrow W$ such that the following diagram commutes
\[
\xymatrix
@C=6pc
@R=3pc
{
	\Lambda^k(V) \ar@{-->}[rd]^{\tilde{f}} & \\
	V \times \cdots \times V \ar[u]_\pi \ar[r]_-{f} & W
}
\]
\end{prop}
\begin{proof}
It is clear how we should define the missing map: $\tilde{f}(v_1 \wedge \cdots \wedge v_k) = f(v_1,\ldots,v_k)$. We just have to show this is well-defined and unique.
\end{proof}

Let us apply \cref{universal property of alternating maps} to the case $W = \mathbb R$. The universal property says that the space of alternating maps $\Alt_k(V)$ is isomorphic to the space of linear maps $\Lambda^k(V) \rightarrow \mathbb R$, therefore we have a natural isomorphism
\begin{equation}
\label{alternating, exterior power isomorphism}
\Alt_k(V) \cong \left( \Lambda^k V \right)^*
\end{equation}

There is a pairing $(\ ,\ ) : \Lambda^k(V) \times \Lambda^k(V^*)$ given by
\[ u = u_1 \wedge \cdots \wedge u_k \]
\[ v^* = v_1^* \wedge \cdots \wedge v_k^* \]
\begin{equation}
\label{exterior power pairing}
(u,v^*) = \det(v_i^*(u_j))
\end{equation}
It is easy to see that this gives a bilinear map and is a non-singular pairing, so if we add in the natural isomorphism from \eqref{alternating, exterior power isomorphism} we get the following three canonical isomorphisms
\begin{equation}
\label{exterior power natural isomorphisms}
\Lambda^k(V^*) \cong \left( \Lambda^k(V) \right)^* \cong \Alt_k(V)
\end{equation}
We can also direct sum all of these isomorphisms to get a sequence of isomorphisms
\[ \Lambda(V^*) \cong \left( \Lambda(V) \right)^* \cong \Alt(V) \]
These isomorphisms allow us to induce an algebra structure on $\Alt_k(V)$ and $\Alt(V)$ from the algebra structure on $\Lambda^k(V^*)$ and $\Lambda(V^*)$ respectively.

Unfortunately there is another pairing we could have chosen instead of \eqref{exterior power pairing}, and is the cause of great confusion in differential geometry. It comes up naturally in certain situations, such as physics and complex geometry, and it is defined by
\begin{equation}
\label{other exterior power pairing}
(u,v^*) = \frac{1}{k!} \det(v_i^*(u_j))
\end{equation}
This pairing is also non-singular, so we get natural isomorphisms as in \eqref{exterior power natural isomorphisms}, except the isomorphisms will of course be different. Since the algebra structure on $\Alt_k(V)$ comes from these isomorphisms we have that each pairing leads to a different algebra structure on $\Alt_k(V)$, so we investigate this now.

Let the multiplication on $\Alt(V)$ induced by the pairing \eqref{exterior power pairing} be denoted by $\wedge_1$, and the multiplication induced by \eqref{other exterior power pairing} be denoted by $\wedge_2$. Let $f \in \Alt_p(V)$ and $g \in \Alt_q(V)$. By the natural isomorphism $\Alt(V) \cong (\Lambda(V))^*$ we can find $v_1^*,\ldots,v_p^* \in V^*$ such that $(-,v_1^*,\ldots,v_p^*) = f$, and similarly with $w_{p+1}^*,\ldots,w_{p+q}^*$ and $g$. Actually, asking for this is a little much since we might really need linear combinations of elements of this form. However, let us assume this special case and the general case will follow from linearity. By definition we now have
\[ f \wedge_1 g(u_1,\ldots,u_{p+q}) = \det\left( \begin{array}{c} v_i^*(u_j) \\ \text{---------} \\ w_i^*(u_j) \end{array} \right) \]
It turns out this is equal to \todo{explain why}
\[ f \wedge_1 g(u_1,\ldots,u_{p+q}) = \sum_{\sigma \in S_{p,q}} (-1)^\sigma f(u_{\sigma(1)},\ldots,\sigma(p)) g(u_{\sigma(p+1)},\ldots,u_{\sigma(p+q)}) \]
where $S_{p,q}$ is the subgroup of $S_{p+q}$ consisting of permutations $\sigma$ such that $\sigma(1)<\cdots<\sigma(p)$ and $\sigma(p+1)<\cdots<\sigma(p+q)$, called the $(p,q)$-shuffles. One can also check that
\[ f \wedge_2 g(u_1,\ldots,u_{p+q}) = \frac{1}{(p+q)!} \sum_{\sigma \in S_{p+q}} (-1)^\sigma f(u_{\sigma(1)},\ldots,\sigma(p)) g(u_{\sigma(p+1)},\ldots,u_{\sigma(p+q)}) \]
It is easy to check that the formulas that these two formulas are related by
\[ f \wedge_1 g = \frac{(p+q)!}{p! q!} f \wedge_2 g \]
We will always use the first pairing \eqref{exterior power pairing}, and so we will just write $\wedge$ instead of $\wedge_1$ when wedging alternating maps. The discussion of these two conventions is necessary because different authors use different pairings, and it can mess up formulas if you are not careful. Choosing this pairing is also related to taking alternations in the following way
\begin{equation}
\label{wedge in terms of alternation}
f \wedge g = \frac{(p+q)!}{p!q!} \Alt(f \otimes g)
\end{equation}
where $\Alt$ is the alternation functor, i.e. for an $r$-multilinear map $T : V \times \cdots \times V \rightarrow \mathbb R$, $\Alt T$ is the alternating, $r$-multilinear map defined by
\[ \Alt T = \frac{1}{r!} \sum_{\sigma \in S_r} (-1)^\sigma T \circ \sigma \]
If we remove the coefficient in \eqref{wedge in terms of alternation} we get the equivalent statement for the second pairing \eqref{other exterior power pairing}. Note that if $f,g \in \Alt_1(V)$, then their wedge is particularly simple
\begin{equation}
\label{simple formula for wedge of 1-forms}
f \wedge g(v,w) = f(v)g(w) - f(w)g(v)
\end{equation}
The choice of pairings also give different formulas for the exterior derivative and interior multiplication. Since we are not dealing with the second pairing we will list these formulas only for the pairing \eqref{exterior power pairing}. 



%%%%%%%%%%%%%%
\note{this does not belong here... move} If $\omega \in \Omega^k(M)$ is a $k$-form, then
\begin{align*}
	d\omega(X_1,\ldots,X_{k+1}) =& \sum_{i=1}^{k+1} (-1)^i X_i \omega(X_1,\ldots,\widehat{X_i},\ldots,X_{k+1}) + \\
	                             & \sum_{1 \leq i < j \leq k+1} (-1)^{i+j} \omega([X_i,X_j],X_1,\ldots,\widehat{X_i},\ldots,\widehat{X_j},\ldots,X_{k+1})
\end{align*}
In particular, if $\omega$ is a 1-form, then we have the simple formula
\begin{equation}
\label{simple formula for d of 1-form}
d\omega(X,Y) = X\omega(Y) - Y\omega(X) - \omega([X,Y])
\end{equation}
%%%%%%%%%%%%%%












\section{Fiber Bundles and Associated Bundles}
\label{Fiber Bundles and Associated Bundles}


\subsection{Fiber Bundles}


A \textbf{fiber bundle} $E$ with base $M$ and fiber $F$ consists of a smooth map $E \stackrel{\pi}{\rightarrow} M$ such that at each point $p \in M$ there is an open set $U$ containing $p$ and a diffeomorphism $\varphi : \pi^{-1}(U) \rightarrow U \times F$ such that $\pi_U \circ \varphi = \pi$, where $\pi_U$ is projection onto the first factor of $U \times F$. This last condition simply says that $\varphi$ is a fiber preserving diffeomorphism. We call $\varphi$ a \textbf{local trivialization}, or sometimes a \textbf{chart}, of the fiber bundle, and when restricted to a fiber $\pi^{-1}(p)$ we get a diffeomorphism $\pi^{-1}(p) \rightarrow F$. So, the fibers are isomorphic to $F$, but not canonically so. We shall use the notation $E_p = \pi^{-1}(p)$ to denote the fiber over a point $p$. An \textbf{atlas} for a fiber bundle is a choice of preferred trivializations, so it consists of an open cover $\lcb U_\alpha \rcb$ and charts $\varphi_\alpha : \pi^{-1}(U_\alpha) \rightarrow U_\alpha \times F$. Any fiber bundle possesses many atlases, some of which may be preferred over others. Given an atlas, consider the composition of two charts:
\[ \varphi_\alpha \circ \varphi_\beta^{-1} : (U_\alpha \cap U_\beta) \times F \rightarrow (U_\alpha \cap U_\beta) \times F \]
Since the charts preserve fibers we must have that this function is of the form $(p,x) \mapsto (p,g_{\alpha\beta}(p)(x))$ for some smooth function $g_{\alpha\beta} : U_\alpha \cap U_\beta \rightarrow \Diff(F)$. These functions are called the \textbf{transition functions} of the atlas. It is easy to see that for any triple of indices $\alpha,\beta,\gamma \in \Lambda$ the transition functions satisfy what is known as the \textbf{cocycle condition}
\begin{equation}
\label{cocycle condition}
g_{\alpha\beta}(p) \circ g_{\beta\gamma}(p) \circ g_{\gamma\alpha}(p) = \id
\end{equation}
for all $p \in U_\alpha \cap U_\beta \cap U_\beta$. The name for this condition comes from its connection with \Cech cohomology. Note that two immediate consequences of the coycle condition are
\begin{align}
	g_{\alpha\alpha}(p) &= \id \\
	g_{\alpha\beta}(p) \circ g_{\beta\alpha}(p) &= \id
\end{align}
For a general fiber bundle we call $\Diff(F)$ the \textbf{structure group} of the fiber bundle, but sometimes the image of the $g_{\alpha\beta}$'s take values in a subgroup of $\Diff(F)$ rather than the full diffeomorphism group. For example, the trivial fiber bundle $M \times F \rightarrow M$ can be covered by a single chart $\varphi = \id : \pi^{-1}(M) \rightarrow M \times F$, and so the transition function (there is only one) from this atlas is the constant map at the identity $g(p) = \id_F$. The image of $g$ is clearly a lot smaller than $\Diff(F)$. In general, if the image of all the $g_{\alpha\beta}$'s lie in a subgroup $G \subseteq \Diff(F)$, then we also say that $G$ is the structure group of the fiber bundle. In general, a fiber bundle can have many structure groups, and the one we call \emph{the} structure group largely depends on what extra structure the bundle carries.


Suppose we only started with a manifold $M$ with open cover $\lcb U_\alpha \rcb_{\alpha \in \Lambda}$, a subgroup $G \subseteq \Diff(F)$, and smooth maps $g_{\alpha\beta} : U_\alpha \cap U_\beta \rightarrow G$ that satisfy \eqref{cocycle condition}. Can we construct a fiber bundle with structure group $G$ and transition functions $\lcb g_{\alpha\beta} \rcb$? Let $\tilde{E}$ be the disjoint union of the spaces $U_\alpha \times F$, as $\alpha$ ranges over $\Lambda$. Put an equivalence relation on $\tilde{E}$ by saying two points $(p,x) \in U_\alpha \times F$ and $(q,y) \in U_\beta \times F$ are equivalent if $p=q$ and $y = g_{\alpha\beta}(p)(x)$. Let $E$ be the quotient space of $\tilde{E}$ by this equivalence relation, and $\pi : E \rightarrow M$ defined by $\pi([p,x]) = p$. Then $E$ is a smooth manifold, $\pi$ is smooth, and $E \stackrel{\pi}{\rightarrow} M$ is a fiber bundle. The atlas of this fiber bundle consist of the local trivializations $\varphi_\alpha : \pi^{-1}(U_\alpha) \rightarrow U_\alpha \times F$ where $\varphi_\alpha([p,x]) = (p,x)$, and clearly the transition functions are $\lcb g_{\alpha\beta} \rcb$. 

If $U$ is an open set of $M$, then we can form the restricted fiber bundle $\pi : \pi^{-1}(U) \rightarrow U$ by looking only at the fibers that lie above $U$. A \textbf{morphism of bundles} $E \stackrel{\pi}{\rightarrow} M$ and $E' \stackrel{\pi'}{\rightarrow} M'$ is a pair of maps $f' : E \rightarrow E'$ and $f : M \rightarrow M'$ such that $f \circ \pi' = \pi \circ f$. Note that $f$ is completely determined by $f'$, so sometimes we will refer to $f'$ as the morphism. Most of the time we will be concerned with the case where $M=M'$ and $f=\id_M$, in which case we require $\pi' = \pi \circ f$. A fiber bundle $F \rightarrow E \rightarrow M$ is said to be \textbf{trivial} if it is isomorphic to the trivial bundle $M \times F \rightarrow M$. If $f : N \rightarrow M$ is a smooth map into the base, we can form the \textbf{pullback} fiber bundle $f^*E \stackrel{\pi'} \rightarrow M$ which is defined by
\[ f^*E = \lcb (x,e) \in N \times E \st f(x) = \pi(e) \rcb \]
\[ \pi'(x,e) = x \]
Then $f^*E$ is a fiber bundle. It is easy to see that an atlas on $E$ with transition functions $\lcb g_{\alpha\beta} \rcb$ induces an atlas on $f^*E$ with transition functions $\lcb g_{\alpha\beta} \circ f \rcb$. 


A section of a fiber bundle is a smooth map $s : M \rightarrow E$ such that $\pi \circ s = \id$, i.e. $s(p)$ is an element of the fiber $E_p$. Not every fiber bundle admits a section. A nice example is to let $E \rightarrow S^1$ be the non-trivial $\mathbb R$-bundle over $S^1$, i.e. $E$ is the open \Mobius band fibered over $S^1$. There is a copy of $S^1$ embedded in $E$ corresponding to the 0 element of $\mathbb R$, the fiber. Let $E'$ be $E$ with this copy of $S^1$ removed, then $E' \rightarrow S^1$ is an $\mathbb R^\times$-bundle. A section of this bundle is simply a smooth choice of component of $\mathbb R^*$ over each point of $S^1$. Such a choice cannot be made smoothly, as can be easily checked by drawing a picture of the \Mobius band. The set of sections of $E$ is denoted by $\Gamma(E)$, and the set of locally defined sections over an open set $U$ is denoted by $\Gamma(U)$. 

A section $s$ of a fiber bundle $E \rightarrow M$ has a local representation in charts. Let $\varphi_\alpha : \pi^{-1}(U_\alpha) \rightarrow U_\alpha \times F$ be a chart, then $\varphi_\alpha \circ s : U_\alpha \rightarrow U_\alpha \times F$ is of the form $p \mapsto (p,s_\alpha(p))$ for some $s_\alpha : U_\alpha \rightarrow F$, called the \textbf{local representation of $s$}. If $\varphi_\beta$ is another chart and $s_\beta$ is the local representation of $s$ in this chart, then $s_\alpha$ and $s_\beta$ are related on $U_\alpha \cap U_\beta$ by
\begin{equation}
\label{section transformation}
s_\alpha(p) = g_{\alpha\beta}(p)(s_\beta(p))
\end{equation}


\subsection{Vector Bundles}


A fiber bundle $F \rightarrow E \rightarrow M$ with transition functions $\lcb g_{\alpha\beta} \rcb$ is said to be a rank $k$ \textbf{real vector bundle} if $F$ is $\mathbb R^k$ and the transition functions take values in $\GL(\mathbb R^k)$. If we replace $\mathbb R$ by $\mathbb C$ or $\mathbb H$ then we get the definition of a rank $k$ \textbf{complex} or \textbf{quaternionic vector bundle} respectively. When $k=1$ we call the vector bundle a (real, complex, quaternionic) line bundle. We can endow each fiber $E_p$ with the structure of a (real, complex, quaternionic) vector space. Let $\mathbb F = \mathbb R,\mathbb C$ or $\mathbb H$, and let $\varphi_\alpha : \pi^{-1}(U_\alpha) \rightarrow U_\alpha \times \mathbb F^k$ be a chart around $p \in M$, then $\varphi_\alpha|_{E_p} : E_p \rightarrow \lcb p \rcb \times \mathbb F^k$ is a linear map. So, for points $v,w \in E_p$ and scalar $c \in \mathbb R, \mathbb C, \mathbb H$ we define 
\[ v+w = \varphi_\alpha^{-1}(\varphi_\alpha(v) + \varphi_\alpha(w)) \]
\[ cv  = \varphi_\alpha^{-1}(c \varphi_\alpha(v)) \]
The fact that the transition functions take values in the linear maps means these operations are well-defined, i.e. does not depend on the chart $\varphi_\alpha$. The trivial vector bundle of rank $k$ over $M$ will be denoted by $\varepsilon^k(M)$, or simply $\varepsilon^k$ when $M$ is clear from context. 

All vector bundles admit many sections, namely the zero section $s(p) = $ the zero vector of $E_p$. A frame for a rank $k$ vector bundle $E \rightarrow M$ is a collection of sections $s_1,\ldots,s_k : M \rightarrow E$ such that $s_1(p),\ldots,s_k(p)$ forms a basis for $E_p$ for all $p \in M$. Only trivial vector bundles admit global frames, but locally  we can always find frames (this is equivalent to local triviality). 

Functors on the category of vector spaces usually lift to functors on the category of vector bundles. Let $F: \Vect \rightarrow \Vect$ a functor on the category of vector spaces. We say that $F$ is a \textbf{$C^k$ functor} (with $k$ possibly $0$ or $\infty$) if for fixed vector spaces $V$ and $W$ the map $\Hom(V,W) \rightarrow \Hom(F(V),F(W))$ defined by $f \mapsto T(f)$ is $C^k$, where $\Hom(V,W)$ is given its natural $C^k$ structure. We can similarly define $C^k$ functors on product categories, i.e. $\Vect \times \cdots \times \Vect \rightarrow \Vect$. 

If $F$ is a $C^\infty$ functor on $\Vect$, then $F$ can be lifted to a functor $F : \Vect(M) \rightarrow \Vect(M)$ on the category of smooth vector bundles over $M$. The structure is to simply let $F(E)$ be the vector bundle obtained by replacing each fiber $E_p$ with $F(E_p)$. The space $F(E)$ inherits a natural manifold structure making $F(E)$ into a smooth vector bundle. 

Using these constructions we can form new vector bundles from old. Let $E,E'$ be two vector bundles over $M$. Then $E \otimes E', E \otimes E'$, $\Hom(E,E')$, $\End(E)$ and $E^* = \Hom(E,\varepsilon^1)$ are new vector bundles obtained by replacing fibers with direct sums, tensor products, homomorphisms, endomorphisms and duals respectively. An important consequence of these constructions is that if two functors $F$ and $G$ are naturally isomorphism, then $F(E)$ and $G(E)$ will be isomorphic as vector bundles. For example, there is a natural isomorphism $V \otimes V^* \cong \End(V)$ for any vector space $V$, hence for any vector bundle $E$ we have $E \otimes E^* \cong \End(E)$. Two other important functors are $\Lambda^k$ and $S^k$, the $k$-th exterior and symmetric power.

We can describe the transition functions of vector bundles obtained from applying functors in terms of the original transition functions. Suppose $E \stackrel{\pi}{\rightarrow} M$ and $E' \stackrel{\pi'}{\rightarrow} M$ have transition functions $g_{\alpha\beta}$ and $g'_{\alpha\beta}$, and let $E \oplus E' \stackrel{\tilde\pi}{\rightarrow} M$ be the direct sum of these vector bundles. If $\varphi_\alpha : \pi^{-1}(U_\alpha) \rightarrow U_\alpha \times \mathbb R^n$ and $\varphi'_\alpha : \pi'^{-1}(U_\alpha) \rightarrow U_\alpha \times \mathbb R^m$ are trivializations, then we can form a trivialization of $E \oplus E'$ by
\[ \tilde\varphi_\alpha : \tilde\pi^{-1}(U_\alpha) \rightarrow \mathbb R^{n+m} \ \ \ \ \ \ \tilde\varphi_\alpha(e,e') = (\varphi_\alpha(e),\varphi'_\alpha(e')) \]
If we take another set of such trivializations $\varphi_\beta,\varphi'_\beta,\tilde\varphi_\beta$ then we see that the transition functions of $E \oplus E'$ are of the form
\begin{align*}
	\tilde\varphi_\alpha \circ \tilde\varphi_\beta^{-1}(p,(v,w)) &= \tilde\varphi_\alpha(\varphi_\beta^{-1}(p,v),\varphi'^{-1}_\beta(p,w)) \\
	                                                             &= (\varphi_\alpha \circ \varphi_\beta^{-1}(p,v), \varphi'_\alpha \circ \varphi'^{-1}_\beta(p,w)) \\
	                                                             &= (g_{\alpha\beta}(p)(v), g'_{\alpha\beta}(p)(w))
\end{align*}
Therefore $E \oplus E'$ has transition functions $g_{\alpha\beta} \oplus g'_{\alpha\beta}$. A similar argument shows that $E \otimes E'$ has transition functions $g_{\alpha\beta} \otimes g'_{\alpha\beta}$.

If we can figure out the transition functions of the dual bundle $E^*$ of $E$, then we automatically get the transition functions for $\Hom(E,E')$ due to the natural isomorphism $V^* \otimes W \cong \Hom(V,W)$ for any vector spaces $V,W$. Let $E^* \stackrel{\tilde\pi}{\rightarrow} M$ be the dual bundle of $E$, and fix an isomorphism $\phi : \mathbb R^n \rightarrow (\mathbb R^n)^*$. Let $\varphi_\alpha : \pi^{-1}(U_\alpha) \rightarrow U_\alpha \times \mathbb R^n$ be a trivialization of $E$. This induces a trivialization $\tilde\varphi_\alpha : \tilde\pi^{-1}(U_\alpha) \rightarrow U_\alpha \times \mathbb R^n$ of $E^*$ in the following way. The restriction of $\varphi_\alpha$ to a fiber $E_p$ is a linear isomorphism $\varphi_\alpha|_{E_p} : \pi^{-1}(p) \rightarrow \lcb p \rcb \times \mathbb R^n$. The dual of $\varphi_\alpha|_{E_p}$ is a linear isomorphism $(\varphi_\alpha|_{E_p})^* : \lcb p \rcb \times (\mathbb R^n)^* \rightarrow (E_p)^* = E_p^*$. This map is going the wrong direction if we want to use it for a trivialization, so we define $\tilde\varphi_\alpha$ restricted to the fiber $E_p^*$ by
\[ \tilde\varphi_\alpha|_{E_p^*} = \phi^{-1} \circ \left( (\varphi_\alpha|_{E_p})^* \right)^{-1} \]
Note that we post-compose with $\phi^{-1}$ so that the range of this function is $\mathbb R^n$ and not $(\mathbb R^n)^*$. Now we have a trivialization $\tilde\varphi_\alpha$ of $E^*$. Let $\varphi_\beta$ be another trivialization of $E$ and $\tilde\varphi_\beta$ the induced trivialization of $E^*$. Then
\begin{align*}
   \tilde\varphi_\alpha \circ \tilde\varphi_\beta^{-1} &= (\id \times \phi^{-1}) \circ (\varphi_\alpha^*)^{-1} \circ \varphi_\beta^* \circ (\id \times \phi)  \\
                                                       &= (\id \times \phi^{-1}) \circ (\varphi_\alpha^{-1})^* \circ \varphi_\beta^* \circ (\id \times \phi) \\
                                                       &= (\id \times \phi^{-1}) \circ (\varphi_\beta \circ \varphi_\alpha^{-1})^* \circ (\id \times \phi) 
\end{align*} 
So the transition functions $\lcb \tilde{g}_{\alpha\beta} \rcb$ of $E^*$ satisfy
\[ \tilde g_{\alpha\beta} = \phi^{-1} \circ g_{\beta\alpha}^* \circ \phi = \phi^{-1} \circ (g_{\alpha\beta}^{-1})^* \circ \phi \]
This is precisely the equation that tells us that the matrix $\tilde g_{\alpha\beta}$ is the inverse of the transpose of $g_{\alpha\beta}$. Therefore the transition functions of the dual bundle are $(g_{\alpha\beta}^t)^{-1}$. 


A \textbf{metric} $g$ on a real vector bundle $E \rightarrow M$ is a section of $S^2 E^* \rightarrow M$ such that for every $p \in M$ the association $X \mapsto g_p(X,-) \in E_p^*$ is an isomorphism. This is equivalent to saying that we have a bilinear, non-degenerate form defined on each fiber $E_p$ that varies smoothly with $p$. It is easy to see that every vector bundle has a metric by patching together local metrics with a partition of unity. If the metric happens to be positive-definite on each fiber, then we may call it a \textbf{positive-definite metric} or \textbf{Riemannian metric}. Otherwise, we may call $g$ an \textbf{indefinite metric} or a \textbf{pseudo-Riemannian metric}, and define its \textbf{signature} to be the signature of $g$ on any fiber, which is the number of positive eigenvalues of $g$. A \textbf{Hermitian metric} on a complex vector bundle $E \rightarrow M$ is a section $g$ of $E^* \otimes E^* \rightarrow M$ such that $g_p : E_p \times E_p \rightarrow \mathbb C$ is a Hermitian inner product, which means that $g_p(-,-)$ is $\mathbb C$-linear in the first variable, conjugate linear in the second variable, satisfies $g_p(v,w) = \overline{g_p(w,v)}$ and $g_p(v,v) \geq 0$ with equality only for $v=0$. 

We say that a real vector bundle $E \rightarrow M$ is orientable if we can choose an orientation for each fiber $E_p$ and find an atlas of $E$ such that each trivialization $\varphi_\alpha : \pi^{-1}(U_\alpha) \rightarrow U \times \mathbb R^n$ is orientation preserving on all fibers, where each $\lcb * \rcb \times \mathbb R^n$ is given the standard orientation. We clearly have that a vector bundle is orientable if and only if there is an atlas whose transition functions takes values in $GL^+(\mathbb R^k)$, the group of matrices of positive determinant. Since complex vector spaces have a natural orientation when regarded as a real vector space we we have that a complex vector bundle is always orientable when regarded as a real vector bundle.

There is another way of looking at orientability of real vector bundles. First we prove two elementary results concerning real line bundles.
\begin{prop}
Let $E \rightarrow M$ be a real line bundle over a connected base and let $\tilde{E}$ be $E$ with the zero section removed so that $\tilde{E}$ is an $\mathbb R^\times$ fiber bundle.
\begin{enumerate}
	\item $E$ is orientable if and only if it is trivial.
	\item $E$ is orientable if and only if $\tilde{E}$ is disconnected.
\end{enumerate}
\end{prop}
\begin{proof}
\sloppyspace
\begin{enumerate}
	\item If $E$ is trivial then it is obvious that $E$ is orientable. Suppose $E$ is orientable and let $g$ be any Riemannian metric on $E$. Since $E$ is orientable we have that there is a chosen component of $E_p-0$ for each fiber. We construct a section $s : M \rightarrow E$ by mapping $p$ to the unique vector in the chosen component of $E_p-0$ with unit length (using the metric). This is a smooth, non-zero section, which gives a global trivialization of $E$ since it is a line bundle.
	
	 \item If $E$ is orientable, then $E$ is trivial by the above, and so $\tilde{E} = M \times \mathbb R^\times$, which is disconnected. Conversely, suppose $\tilde{E}$ is disconnected and let $g$ be a Riemannian metric on $E$. Let $\tilde{E}_+$ be a component of $\tilde{E}$, then define a section $s : M \rightarrow E$ by mapping $p$ to the unique vector in the of $\tilde{E}_+$ over $p$ of unit length. This is a smooth, non-zero section, and so $E$ is trivial and hence orientable.
\end{enumerate}
\end{proof}

Applying the functor $\Lambda^k$ to our vector bundle $E$, where $k$ is the rank of $E$, produces a one-dimensional vector bundle, called the \textbf{determinant line bundle} of $E$. If we remove the zero section, $\Lambda^k E - 0$, then each fiber has two components, and a choice of one of the components is equivalent to choosing an orientation on that fiber. Choosing a component of these fibers in a smooth manner is equivalent to find a smooth, non-zero section, hence $\Lambda^k E$ must be the trivial line bundle. Conversely, if $\Lambda^k(V)$ is trivial, then we can find a smooth non-zero section, and so we can choose orientations on fibers in a smooth manner. We have proved the following.
\begin{prop}
\label{equivalent statements of orientability}
The follow statements are equivalent to a vector bundle $E \rightarrow M$ being orientable.
\begin{enumerate}
	\item There is an atlas for $E$ whose transition functions take values in $GL^+(\mathbb R^k)$.
	\item The determinant line bundle $\Lambda^k E$ is trivial.
	\label{determinant line bundle is trivial}
	\item The determinant line bundle minus the zero section, $\Lambda^k E - 0$, consists of exactly two components.
	\label{determinant line bundle minus zero section is disconnected}
\end{enumerate}
\end{prop}
Further, we can interpret a choice of non-zero section in \ref{determinant line bundle is trivial} of \cref{equivalent statements of orientability} as a choice of orientation for $E$, and a choice of component of $\Lambda^k E-0$ in \ref{determinant line bundle minus zero section is disconnected} of \ref{equivalent statements of orientability} as a choice of orientation of $E$. Therefore every orientable vector bundle has precisely two orientations.

We have just seen that choosing an orientation on a vector bundle is equivalent to ``reducing'' the structure group from the full linear group $GL(\mathbb R^k)$ to just those matrices with positive determinant $GL^+(\mathbb R^k)$. This reduction of structure group comes up often, so we explore it more now. Suppose we put a metric $g$ of signature $s$ on $E$. We can choose our atlas so that the trivializations $\varphi_\alpha : \pi^{-1}(U_\alpha) \rightarrow U_\alpha \times \mathbb R^k$, when restricted to fibers, preserve the inner product. Specifically, we should have
\[ g_p(v,w) = \< \varphi_\alpha(v),\varphi_\alpha(w) \> \]
for all $v,w \in E_p$, where $\<\ ,\ \>$ is the standard inner product on $\mathbb R^k$ of signature $s$. With these trivializations it is clear that the transition functions will take values in $O(s,k-s)$. In particular, introducing a Riemannian metric on $E$ is equivalent to reducing the structure group to $O(k)$, and since every vector bundle supports a metric we can always reduce to $O(k)$. Combining this with our results on orientability we see that an orientable vector bundle can have its structure group reduced to $SO(k)$. Similarly, introducing a Hermitian metric on a complex vector bundle is equivalent to reducing the structure group from $GL(\mathbb C^k)$ to $U(k)$. 


\todo{abstract way of defining reduction of structure group}



\subsection{Principal $G$-Bundles}



Given a Lie group $G$, a principal $G$-bundle is a fiber bundle $G \rightarrow P \stackrel{\pi}{\rightarrow} M$ such that $G$ acts freely and transitively on the right of $P$ and preserves the fibers. This means the fibers of $P$ are principal homogeneous spaces of $G$, hence they are diffeomorphic to $G$. A chart of a principal $G$-bundle is a fiber preserving, $G$-equivariant diffeomorphism $\varphi_\alpha : \pi^{-1}(U_\alpha) \rightarrow U_\alpha \times G$, which means $\varphi_\alpha(x \cdot g) = \varphi_\alpha(x) \cdot g$, where the action of $G$ on $U_\alpha \times G$ is just right-multiplication on the second factor. An atlas for a principal bundle is a collection of such charts where the $U_\alpha$'s cover the base. 

An atlas for $P \rightarrow M$ leads to a collection of transition functions $\lcb h_{\alpha\beta} : U_\alpha \cap U_\beta \rightarrow \Diff(G) \rcb$ as before. However, due to the $G$-equivariance of charts, these transition functions induce smooth maps $g_{\alpha\beta} : U_\alpha \cap U_\beta \rightarrow G$ that satisfy a cocycle-like condition
\begin{equation}
\label{principal cocycle condition}
g_{\alpha\beta}(p) \cdot g_{\beta\gamma}(p) \cdot g_{\gamma\alpha}(p) = 1
\end{equation}
where $\cdot$ denotes the group multiplication in $G$, and 1 is the identity in $G$. These maps are defined in the following way. For two charts $\varphi_\alpha,\varphi_\beta$ we have
\begin{align*}
	(p,h_{\alpha\beta}(p)(g)) &= \varphi_\alpha(\varphi_\beta^{-1}(p,g)) \\
	                          &= \varphi_\alpha(\varphi_\beta^{-1}((p,1)\cdot g)) \\
	                          &= \varphi_\alpha((\varphi_\beta^{-1}(p,1)) \cdot g) \\
	                          &= \varphi_\alpha(\varphi_\beta^{-1}(p,1)) \cdot g \\
	                          &= (p, h_{\alpha\beta}(p)(1) \cdot g)
\end{align*}
which implies $h_{\alpha\beta}(p)(g) = h_{\alpha\beta}(p)(1) \cdot g$. So we set $g_{\alpha\beta}(p) = h_{\alpha\beta}(p)(1)$, and these $g_{\alpha\beta}$'s clearly satisfy the cocycle condition \eqref{principal cocycle condition}. The maps $g_{\alpha\beta}$ are called \textbf{principal transition functions} for the atlas of $P$. Conversely, suppose we are given an open covering $\lcb U_\alpha \rcb$ and a collection of smooth maps $g_{\alpha\beta} : U_\alpha \cap U_\beta \rightarrow G$ that satisfy \eqref{principal cocycle condition}, then we can create a principal $G$-bundle with principal transition functions $\lcb g_{\alpha\beta} \rcb$ in the following way. Let $\tilde{P}$ be the disjoint union of $U_\alpha \times G$ as $\alpha$ runs over the index set of our cover. Say two elements $(p,g) \in U_\alpha \times G$ and $(q,h) \in U_\beta \times G$ are equivalent if $p=q$ and $g = g_{\alpha\beta}(p) \cdot h$ (where $\cdot$ is group multiplication), and let $P$ be the quotient of $\tilde P$ by this relation. The right action of $G$ on $P$ is given by $[p,g] \cdot g' = [p,gg']$. With this action we have that the obvious charts $\varphi_\alpha([p,g]) = (p,g)$ are $G$-equivariant since $\varphi_\alpha([p,g] \cdot g') = \varphi_\alpha([p,gg']) = (p,gg')=(p,g) \cdot g'$. The obvious projection $P \rightarrow M$ makes $P$ into a principal $G$-bundle. 

This seemingly esoteric reworking of the definition of principal bundles is actually quite enlightening. It tells us that principal $G$-bundles are the special class of $G$ fiber bundles whose transition functions can be taken to be left-multiplication by a group element:
\[ \varphi_\alpha \circ \varphi_\beta^{-1}(p,g) = (p, g_{\alpha\beta}(p) \cdot g) \]


A morphism of principal $G$-bundles $P \rightarrow M$ and $P' \rightarrow M'$ is a morphism of fiber bundles $(f',f)$, defined before, that is also $G$-equivariant, i.e. $f(x \cdot g) = f(x) \cdot g$ for all $x \in P$. Most of the time we will take $M' = M$ and $f = \id_M$. Also the pullback of a principal $G$-bundle is a principal $G$-bundle. 


The existence of a global section of a principal $G$-bundle turns out to be quite restrictive.
\begin{prop}
\label{global section trivial principal bundle}
A principal $G$-bundle is trivial if and only if it admits a section.
\end{prop}
\begin{proof}
One direction is obvious. Suppose $P \rightarrow M$ is a principal $G$-bundle, and let $s : M \rightarrow P$ be a section. We define a bundle map $\tilde{f} : P \rightarrow M \times G$ in the following way. For $u \in P$, let $g$ be the unique element in $G$ such that $ug = s(\pi(u))g$. Existence of such an element is guaranteed by the transitivity of the action of $G$, and uniqueness is guaranteed by the freeness of the action. Then define $\tilde{f}(u) = (\pi(u),g)$. This map is clearly smooth, bijective, and a bundle map, and so it follows that it is an equivalence of bundles.
\end{proof}
\begin{cor}
If $P \stackrel{\pi}{\rightarrow} M$ is a principal $G$-bundle, then $\pi^* P$ is trivial (the pullback bundle along $\pi$).
\end{cor}
\begin{proof}
The total space of the pullback is $\pi^* P = \lcb (p,q) \in P \times P \st \pi(p) = \pi(q) \rcb$, that is, it consists of pairs of points that lie in the same fiber. There is a global section $s : P \rightarrow \pi^* P$ of this bundle by setting $s(p) = (p,p)$. By \cref{global section trivial principal bundle} we have that $\pi^* P$ is trivial.
\end{proof}




\subsection{Associated Fiber Bundles}


There is a construction that allows one to create many fiber bundles out of a fixed bundle. Let $E \rightarrow M$ be a fiber bundle with fiber $F$, structure group $G \subseteq \Diff(F)$, and equipped with an atlas with transition functions are $\lcb g_{\alpha\beta} \rcb$. Suppose $G$ acts effectively on a space $F'$, that is, there is a smooth injective, homomorphism $\rho : G \rightarrow \Diff(F')$. The collection of functions $\lcb \rho \circ g_{\alpha\beta} \rcb$ satisfy the cocycle condition \eqref{cocycle condition}, and so they determine a fiber bundle $E' \rightarrow M$ with fiber $F'$ and structure group $G$ (since the action is effective), which is called the \textbf{associated fiber bundle} to $E$. If $E$ is a vector bundle with structure group $G$ and $V$ a vector space. Let $\rho : G \rightarrow GL(V)$ be a faithful linear representation of $G$ on $V$. Then the associated fiber bundle $E'$ is a vector bundle called the \textbf{associated vector bundle}. 

There is an invariant way of describing associated bundles when applied to principal $G$-bundles. Let $P \stackrel{\pi}{\rightarrow} M$ be a principal $G$-bundle, and let $\rho : G \rightarrow \Diff(F)$ be an effective left action of $G$ on $F$. Then $G$ acts on the right of the product $P \times F$ via
\begin{equation}
\label{associated bundle action}
(x,f) \cdot g = (x \cdot g, \rho(g^{-1})(f))
\end{equation}
Let $E = P \times_\rho F$ denote the quotient of $P \times F$ by this action, and define $\pi' : E \rightarrow M$ by $\pi'([x,f]) = \pi(x)$. Then $E \stackrel{\pi'}{\rightarrow} M$ is a fiber bundle with fiber $F$, structure group $\image \rho$ (which is isomorphic to $G$ since the action is effective), called the \textbf{associated fiber bundle} to $P \rightarrow M$. If $P$ has principal transition functions $\lcb g_{\alpha\beta} \rcb$, then $E$ has transition functions $\lcb \rho \circ g_{\alpha\beta} \rcb$. Most times we will apply this construction to a vector space $F=V$ and a linear representation $\rho : G \rightarrow GL(V)$, and so the associated bundle will be called the \textbf{associated vector bundle} to $P \rightarrow M$.


A popular example of these constructions is to consider the adjoint action of a Lie group $G$ on itself, $\Ad : G \rightarrow \Aut(G)$ defined by $\Ad(g) = C_g$, where $C_g(h) = ghg^{-1}$. The associated fiber bundle $P \times_{\Ad} G$ is denoted by $\Ad(P)$, and has fiber $G$ and structure group $G$ but is not necessarily a principal $G$-bundle. Related to this, let $\ad : G \rightarrow \GL(\mathfrak g)$ denote the adjoint representation of $G$, i.e. $\ad(g) = (C_g)_* : \mathfrak g \rightarrow \mathfrak g$. The associated vector bundle $P \times_{\ad} \mathfrak g$ is denoted by $\ad(P)$. 







\section{Bundle-Valued Differential Forms}


When working with connections and curvature in general vector bundles, one quickly realizes that a certain generalization of differential forms makes things nicer. We will assume the basic theory of differential forms is known.

Let $E \stackrel{\pi}{\rightarrow} M$ be a vector bundle. A differential $p$-form $\omega$ with values in the vector bundle $E$ will be a section of the bundle $\Lambda^p T^* M \otimes E \rightarrow M$, where $\Lambda^p T^* M$ is the $p$-th exterior power of the cotangent bundle. The $C^\infty(M)$-module of $E$-valued differential $p$-forms will be denoted by $\Omega^p(M;E)$, and we will use the convention that $\Omega^0(M;E)$ is just sections of $E$. There is a natural isomorphism of $\Omega^p(M;E)$ with alternating, $C^\infty(M)$-multilinear maps
\begin{equation}
\label{characterization of forms}
\Gamma(TM) \times \cdots \times \Gamma(TM) \rightarrow \Gamma(E)
\end{equation}
as $C^\infty(M)$-modules. Elements of $\Omega^p(M;E)$ can be written as linear combinations of elements of the form $\omega \otimes s$, where $\omega \in \Omega^p(M)$ and $s \in \Gamma(E)$. This is because for any vector bundles $E_1,E_2$ over $M$ we have an isomorphism of $C^\infty(M)$-modules
\[ \Gamma(E_1 \otimes E_2) \cong \Gamma(E_1) \otimes_{C^\infty(M)} \Gamma(E_2) \]

If the bundle $E$ is trivial, i.e. $E = M \times V$ for some vector space $V$, then we call forms with values in this bundle $V$-valued forms, and denote them by $\Omega^k(M;V)$. As a special case, if $E$ is just the trivial line bundle, $E = M \times \mathbb R$, then an $E$-valued differential form is the same as a regular differential form. 

Most everything one does with regular differential forms goes through to bundle-valued forms with very little changes. For example, if $E_1,E_2$ are vector bundles over $M$, then we can define the wedge product of $E_1$-valued and $E_2$-valued differential forms. In particular, if $\omega \in \Omega^p(M;E_1)$ and $\eta \in \Omega^q(M;E_2)$, then we define $\omega \wedge \eta \in \Omega^{p+q}(M;E_1 \otimes E_2)$ by
\[ \omega \wedge \eta(v_1, \ldots, v_{p+q}) = \sum_{\sigma \in S_{p,q}} (-1)^\sigma \omega(v_{\sigma(1)},\ldots,v_{\sigma(p)}) \otimes \eta(v_{\sigma(p+1)},\ldots,v_{\sigma(p+q)}) \]
All of the usual properties of wedge product follow easily, such as graded commutativity. 

However, the exterior derivative does not always carry over naturally to bundle-valued differential forms. If the vector bundle is flat (to be defined later), then we can define the exterior derivative and find that it has all of the same properties as the regular exterior derivative. For example, if $\omega$ is a $V$-valued differential form, where $V$ is a vector space, then $\omega$ can be written as $\omega = \omega^i \otimes v_i$, where $v_i$ is a basis for $V$ (and also thought of as constant sections) and $\omega^i$ is a regular differential form. Then we define
\[ d \omega = \left( d\omega^i \right) \otimes v_i \]
One can show that this does not depend on how we write $\omega$ as a linear combination of simple elements $\omega^i \otimes v_i$. This exterior differential is a map $\Omega^k(M;V) \rightarrow \Omega^{k+1}(M;V)$, and satisfies $d \circ d = 0$. Using our extension of the wedge product to $V$-valued forms, we also have the property that
\[ d(\omega \wedge \eta) = d\omega \wedge \eta + (-1)^k \omega \wedge d\eta \]
However, in the general case, when the vector bundle is not necessarily trivial, we will need to choose a connection before we can define anything that resembles the exterior derivative. 

There is a special case of bundle-valued forms for when we take forms with values in a Lie algebra $\mathfrak g$. In this case we can define one extra operation. Suppose $\mathfrak g$ is a Lie algebra, and $\omega \in \Omega^p(M;\mathfrak g)$ and $\eta \in \Omega^q(M;\mathfrak g)$. Then we define their bracket $[\omega,\eta]$ to be the composition
\[ \Omega^p(M;\mathfrak g) \times \Omega^q(M;\mathfrak g) \stackrel{\wedge}{\longrightarrow} \Omega^{p+q}(M;\mathfrak g \otimes \mathfrak g) \stackrel{[,]}{\longrightarrow} \Omega^{p+q}(M;\mathfrak g) \]
where the first map is the wedge product for $\mathfrak g$-valued forms, and the second map is induced by the bracket in $\mathfrak g$. It is easy to see that this operation satisfies the following:
\begin{equation}
[\omega,\eta] = (-1)^{pq+1}[\eta,\omega]
\end{equation}
\begin{equation}
d[\omega,\eta] = [d\omega,\eta] + (-1)^p [\omega,d\eta] \\
\end{equation}
\begin{equation}
[[\omega,\omega],\omega] = 0
\end{equation}
If $\omega \in \Omega^1(M;\mathfrak g)$, then we have a simple formula
\begin{equation}
\label{bracket of two 1-forms}
\begin{array}{rcl}
	[\omega,\omega](X,Y) &=& [\omega \wedge \omega(X,Y)] \\
	                     &=& [\omega(X) \otimes \omega(Y) - \omega(Y) \otimes \omega(X)] \\
	                     &=& [\omega(X),\omega(Y)] - [\omega(Y),\omega(X)] \\
	                     &=& 2[\omega(X),\omega(Y)]
\end{array}
\end{equation}
since the bracket in $\mathfrak g$ is antisymmetric.

Let $G$ be a Lie group and $\mathfrak g$ its Lie algebra. Recall that we identify $\mathfrak g$ with left-invariant vector fields on $G$, with the bracket in $\mathfrak g$ given by the regular Lie bracket of vector fields. If $X_1,\ldots,X_n$ are a basis for $\mathfrak g$, then we can uniquely write
\begin{equation}
\label{structure constants}
[X_i,X_j] = c_{ij}^k X_k
\end{equation}
The constants $c_{ij}^k$ are called the \textbf{structure constants} of $\mathfrak g$ with respect to this basis. A left-invariant 1-form on $G$ is a 1-form $\omega \in \Omega^1(G)$ such that $L_g^* \omega = \omega$, where $L_g$ is left translation by $g$. Recall that the dual space $\mathfrak g^*$ can be identified with the set of left-invariant 1-forms on $G$. 

For any $\omega \in \mathfrak g^*$ and $X,Y \in \mathfrak g$ we see that $\omega(X)$ and $\omega(Y)$ are constant as maps $G \rightarrow \mathbb R$ (by the left-invariance of everything). This clearly implies that $X\omega(Y) = Y \omega(X) = 0$, so the popular formula for the exterior differential of a 1-form \eqref{simple formula for d of 1-form}, when acting on left-invariant vector fields, reduces to 
\begin{equation}
\label{d of left-invariant 1-form}
d\omega(X,Y) = -\omega([X,Y])
\end{equation}
Because the tangent bundle of a Lie group is trivial and $d\omega$ is tensorial, knowing this formula on left invariant vector fields implies that it holds on all vector fields. Therefore \eqref{d of left-invariant 1-form} holds for all vector fields $X,Y$, not just those vector fields in $\mathfrak g$.

Let $\theta^1,\ldots,\theta^n$ be the dual basis to $X_1,\ldots,X_n$. We are going to show that $d\theta_i$ can be expressed in terms of the structure constants. Keeping the structure constants \eqref{structure constants} in mind and using \eqref{d of left-invariant 1-form} for a dual basis element $\theta^i$ acting on basis elements $X_j,X_k$ we get
\begin{align*}
	d\theta^i(X_j,X_k) &= -\theta^i\left( [X_j,X_k] \right) \\
	                   &= -\theta^i \left( c_{jk}^\ell X_\ell \right) \\
	                   &= -c_{jk}^i
\end{align*}
The following 2-form acts the same on $X_j,X_k$:
\begin{align*}
	-\frac{1}{2} c_{\ell m}^i \theta^\ell \wedge \theta^m (X_j,X_k) &= -\frac{1}{2} c_{\ell m}^i ( \theta_\ell(X_j)\theta_m(X_k) - \theta_\ell(X_k)\theta_m(X_j) ) \\
	                                                    &= -\frac{1}{2} ( c_{jk}^i - c_{kj}^i ) \\
	                                                    &= -c_{jk}^i
\end{align*}
where the last line follows from the fact $c_{jk}^i = -c_{kj}^i$. Therefore $d\theta^i$ can be expressed in terms of the structure constants by
\begin{equation}
\label{Maurer-Cartan equations in coordinates}
d\theta^i = -\frac{1}{2} c_{jk}^i \theta^j \wedge \theta^k
\end{equation}
These equations are called the \textbf{Maurer-Cartan equations} for $G$ with respect to the basis $X_1,\ldots,X_n$. These equations can be interpreted as an integrability condition in the sense of Frobenius. 

With the basis $X_1,\ldots,X_n$ and dual basis $\theta^1,\ldots,\theta^n$ define the $\mathfrak g$-valued 1-form $\omega_0 = \theta^i X_i$. Note that we should technically write $\theta^i \otimes X_i$ for a $\mathfrak g$-valued 1-form, but this tensor product is a nuisance to write. Just remember that if we juxtapose a vector and its dual, such as $\theta^i X_i$, then we mean their tensor product, and if we write $\theta^i(X_j)$ we mean $\theta^i$ acting on $X_j$. Surprisingly this form does not depend on the basis. For suppose $X = a^i X_i$, then 
\begin{align*}
	\omega_0(X) &= \omega_0(a^i X_i) \\
	            &= \theta^j(a^i X_i) X_j \\
	            &= a^i \theta^j(X_i) X_j \\
	            &= a^i X_i = X
\end{align*}
Therefore $\omega_0(X)=X$ for all $X \in \mathfrak g$, and this uniquely determines $\omega_0$. This distinguished 1-form on a Lie group $G$ is called the \textbf{Maurer-Cartan form} of $G$. It satisfies a Maurer-Cartan type equation. Namely, we see that
\begin{align*}
	d\omega_0 &= d\theta^i X_i \\
	        &= -\frac{1}{2} c_{jk}^i \theta^j \wedge \theta^k X_i \\
	        &= -\frac{1}{2} [\theta^i (X_i),\theta^i (X_i)] \\
	        &= -\frac{1}{2} [\omega_0,\omega_0]
\end{align*}
Therefore $\omega_0$ satisfies
\begin{equation}
\label{Maurer-Cartan equation for the Maurer-Cartan form}
d\omega_0 + \frac{1}{2} [\omega_0,\omega_0] = 0
\end{equation}










\section{Hodge Theory}
\label{Hodge Theory}




We will now apply the ideas of the Hodge start to differential forms. A \textbf{pseudo-Riemannian} manifold is a smooth $n$-manifold $M$ with a metric $g$ on its tangent bundle. If the metric is positive-definite, then $(M,g)$ is simply called a \textbf{Riemannian} manifold. Let $(M,g)$ be a compact, smooth, pseudo-Riemannian manifold, and let $s$ denote the signature of $g$. In coordinates $x : U \rightarrow \mathbb R^n$ the metric can be written as
\[ g = g_{ij} dx^i \otimes dx^k \]
where $(g_{ij})$ is a symmetric, non-degenerate matrix of signature $s$. 

The metric gives us a canonical identification of tangent spaces with cotangent spaces, via $X \mapsto g(X,-)$, which can be used for index raising and lowering, as before. Since the constructions are the same as in \cref{Some Linear Algebra} we will not spell out the details. For a vector $X$, its associated covector will be denoted by $X^\flat$ (lowering the index), and for a covector $\omega$, its associated vector will be denoted by $\omega^\sharp$ (raising the index). We can apply the $\flat$ and $\sharp$ operators to any tensor to lower and raise any index.

A metric $g$ on the tangent bundle of a manifold naturally induces a metric on all of the vector bundles we can derive from $TM$. Again, we will not spell out the details since it is very similar to before, but we will introduce the notation $T_{r,s}(TM) \rightarrow M$ for the tensor bundle of type $(r,s)$ over $M$, i.e. 
\[ T_{r,s}(TM) = \underbrace{TM \otimes \cdots \otimes TM}_{r} \otimes \underbrace{T^*M \otimes \cdots \otimes T^*M}_{s}  \]

A pseudo-Riemannian $n$-manifold $(M,g)$ carries a distinguished $n$-form, called the \textbf{volume form}, which we will usually denote by $\omega$, or $\omega_g$ to show its dependence on $g$. In coordinates it is written as
\[ \omega = \sqrt{|\det (g_{ij})|} \, dx^1 \wedge \cdots \wedge dx^n \]
When $M$ is compact, orientable and $g$ is positive definite, $[\omega]$ will be a generator of $H_{dR}^n(M) = \mathbb R$. 





















\section{Connections on Vector Bundles and Curvature}

We want to get some invariants (hopefully numerical and computable) of vector bundles that will help us get information on the global ``twistedness'' of the vector bundle. To this end we define a connection. Let $E \stackrel{\pi}{\rightarrow} M$ be a vector bundle. A connection $\nabla$ on $E$ is an $\mathbb R$-bilinear map
\[ \nabla : \Gamma(TM) \times \Gamma(E) \rightarrow \Gamma(E) \]
that is $C^\infty(M)$-linear in the first variable and satisfies a Leibniz rule in the second variable. This means that the following two conditions are met:
\begin{equation}
\nabla_{f X} s = f \nabla_X s
\end{equation}
\begin{equation}
\nabla_X (fs) = f \nabla_X s + (X f)s 
\end{equation}
We think of $\nabla$ as allowing us to differentiate sections of $E$ in the direction of vector fields on $M$. It is easy to see that an equivalent definition of connection is an $\mathbb R$-linear map
\[ \nabla : \Gamma(E) \rightarrow \Gamma(T^* M \otimes E) = \Omega^1(M;E) \]
that satisfies the Leibniz rule
\[ \nabla(fs) = f \nabla s + df \otimes s \]

Consider the trivial vector bundle $E = M \times \mathbb R^k$. Sections of this bundle are simply $\mathbb R^k$-valued functions on $M$, and so can be written as a column vector $(s^i)$, where $s^i \in C^\infty(M)$. The trivial connection $\tilde{\nabla}$ on this bundle is simply the exterior derivative:
\[ \tilde{\nabla}_X s = \tilde{\nabla}_X \begin{pmatrix} s_1 \\ \vdots \\ s_k \end{pmatrix} = \begin{pmatrix} ds_1(X) \\ \vdots \\ ds_k(X) \end{pmatrix} \]
or simply written as $\tilde{\nabla} s = ds$, where $d$ is understood to act component-wise. Let $\omega \in \Omega^1(M;\gl(\mathbb R^k))$ be a $\gl(\mathbb R^k)$-valued 1-form, i.e. can be written as $\omega = (\omega_i^j)$, where $\omega_i^j$ are regular 1-forms. Then $d+\omega$ is also a connection on $E \rightarrow M$. It is clear that $d+\omega$ is $C^\infty(M)$ in the first variable. That it is Leibniz in the second comes from quick calculuation:
\begin{align*}
	(d+\omega)_X (fs) &= d(fs)(X) + \omega(X) \cdot fs \\
	                  &= (df \cdot s + f \cdot ds)(X) + f \omega(X) s \\
	                  &= df(X) \cdot s + f c\cdot ds(X) + f \omega(X) s \\
	                  &= df(X) \cdot s + f (d + \omega)(X) s
\end{align*}
It turns out that all connections on a trivial bundle are of this form.
\begin{prop}
Every connection $\nabla$ on a trivial vector bundle $E = M \times R^k$ is of the form $\nabla = d + \omega$, where $\omega \in \Omega^1(M; \gl(\mathbb R^k))$. 
\end{prop}
\begin{proof}
Let $\nabla$ be any connection on $E$, and let $e_i : M \rightarrow \mathbb R^k$ be the constant, standard basis sections, $i = 1, \ldots, k$. Letting $\nabla$ act on these sections we see that for any $X$,
\[ \nabla_X e_i = \begin{pmatrix} \omega_i^1(X) \\ \vdots \\ \omega_i^k(X) \end{pmatrix} = \omega_i^j(X) e_j \]
for some 1-forms $\omega_i^j$. So, if an arbitrary section $s = s^i e_i$ is expressed in the standard basis we have
\begin{align*}
	\nabla_X s &= \nabla_X s^i e_i \\
	           &= ds^i(X) e_i + s^i \nabla_X e_i \\
	           &= ds^i(X) e_i + s^i \omega_i^j(X) e_j 
\end{align*}
So, if we let $\omega = (\omega_i^j)$ be the matrix of 1-forms, then this last line is precisely $d+\omega$ applied to $s$.
\end{proof}

Since every vector bundle is locally trivial, we can choose a connection on each trivialization and patch them together using a partition of unity to get a globally defined connection. Therefore every vector bundle has a connection. Given a connection $\nabla$ on $E \stackrel{\pi}{\rightarrow} M$ and local chart $\varphi_\alpha : \pi^{-1}(U_\alpha) \rightarrow U_\alpha \times \mathbb R^k$, we get a local expression for $\nabla$, which is simply $(\nabla_X s)_\alpha = ds_\alpha + \omega_\alpha s_\alpha$, for some $\gl(\mathbb R^k)$-valued 1-form $\omega_\alpha$, called the local connection 1-form. Suppose we have two local expressions of the connection $\nabla_\alpha$ and $\nabla_\beta$, corresponding to two charts $\varphi_\alpha$ and $\varphi_\beta$ respectively. Since sections transform by \eqref{section transformation}, we must have $d\sigma_\alpha + \omega_\alpha s_\alpha = g_{\alpha\beta} (d \sigma_\beta + \omega_\beta s_\beta)$. So, using \eqref{section transformation} we get
\begin{align*}
	d\sigma_\alpha + \omega_\alpha &= g_{\alpha\beta} (d \sigma_\beta + \omega_\beta s_\beta) \\
	                               &= g_{\alpha\beta} \left( d (g_{\alpha\beta}^{-1} \sigma_\alpha + \omega_\beta g_{\alpha\beta}^{-1} s_\alpha \right) \\
	                               &= g_{\alpha\beta} \left( dg_{\alpha\beta}^{-1} \sigma_\alpha + g_{\alpha\beta}^{-1} d\sigma_\alpha + \omega_\beta g_{\alpha\beta}^{-1} s_\alpha \right) \\
	                               &= g_{\alpha\beta} dg_{\alpha\beta}^{-1} \sigma_\alpha + d\sigma_\alpha + g_{\alpha\beta} \omega_\beta g_{\alpha\beta}^{-1} s_{\alpha} 
\end{align*}
Therefore $\omega_\alpha$ and $\omega_\beta$ are related by
\begin{equation}
\label{connection 1-form transformation}
\begin{array}{rcl}
	\omega_\alpha &=& g_{\alpha\beta} dg_{\alpha\beta}^{-1} + g_{\alpha\beta} \omega_\beta g_{\alpha\beta}^{-1} \\
	\omega_\beta  &=& g_{\alpha\beta}^{-1} dg_{\alpha\beta} + g_{\alpha\beta}^{-1} \omega_\alpha g_{\alpha\beta}
\end{array}
\end{equation}

This leads to an alternate definition of connection. A connection on a vector bundle $E \rightarrow M$ with transition functions $\lcb g_{\alpha\beta} \rcb$ is a collection of differential operators $\lcb d + \omega_\alpha \rcb$ such that the $\omega_\alpha$'s transform like \eqref{connection 1-form transformation}. 

A vector bundle with connection can pullback to a vector bundle with connection. If $\nabla$ is a connection on $E \rightarrow M$ given by $\lcb d + \omega_\alpha \rcb$, then $f^*\nabla$ is the connection on $f^*E \rightarrow N$ given by $\lcb d + f^*\omega_\alpha \rcb$, where $f^*\omega_\alpha$ is simply the component-wise pullback. It is easy to see that this collection transforms like \eqref{connection 1-form transformation} (this is mostly due to the fact that the pullback of a 0-form is just precomposition)
\begin{align*}
	f^*\omega_\alpha &= f^* \left( g_{\alpha\beta} dg_{\alpha\beta}^{-1} + g_{\alpha\beta} \omega_\beta g_{\alpha\beta}^{-1} \right) \\
	                 &= (g_{\alpha\beta} \circ f) \cdot d(g_{\alpha\beta} \circ f)^{-1} + (g_{\alpha\beta} \circ f) \cdot f^*\omega_\beta \cdot (g_{\alpha\beta} \circ f)^{-1}
\end{align*}

Intuitively we now want to define the curvature of a connection $\nabla$ on $E \rightarrow M$ as the ``derivative'' of $\nabla$. This means that if $\lcb d + \omega_\alpha \rcb$ are the local expressions of $\nabla$, then we define a collection $\gl(\mathbb R^k)$-valued 2-forms $\Omega_\alpha$ by
\begin{align*}
\Omega_\alpha s_\alpha &= \nabla \circ \nabla s_\alpha \\
                       &= (d+\omega_\alpha)(ds_\alpha + \omega_\alpha s_\alpha) \\
                       &= d^2 s_\alpha + d(\omega_\alpha s_\alpha) + \omega_\alpha d\omega_\alpha + \omega_\alpha \wedge \omega_\alpha s_\alpha \\
                       &= d\omega_\alpha s_\alpha - \omega_\alpha ds_\alpha + \omega_\alpha d\sigma_\alpha + \omega_\alpha \wedge \omega_\alpha s_\alpha \\
                       &= d\omega_\alpha + \omega_\alpha \wedge \omega_\alpha
\end{align*}
The collection of 2-forms $\lcb \Omega_\alpha \rcb$ is called the curvature of $\nabla$, and the equation
\begin{equation}
\label{structure equation}
\Omega_\alpha = d\omega_\alpha + \omega_\alpha \wedge \omega_\alpha
\end{equation}
is called the structure equation. The transformation from an $\Omega_\alpha$ to an $\Omega_\beta$ is given by
\begin{equation}
\label{curvature 2-form transformation}
\Omega_\alpha = g_{\alpha\beta} \Omega_\beta g_{\alpha\beta}^{-1}
\end{equation}
which tells us that the $\Omega_\alpha$'s actually patch together to form a global $\End(E)$-valued 2-form, which we denote by $\Omega$. The local expressions for the curvature satisfy the \textbf{Bianchi Identity}:
\[ d\Omega_\alpha = \Omega_\alpha \wedge \omega_\alpha - \omega_\alpha \wedge \Omega_\alpha = [\Omega_\alpha,\omega_\alpha] \]
There is a more invariant definition of the curvature of a connection, but we will not use it in what follows. But, for completeness, the curvature of $\nabla$ can be defined to be the map $F^\nabla : \Gamma(TM) \times \Gamma(TM) \times \Gamma(E) \rightarrow \Gamma(E)$ given by
\[ F^\nabla(X,Y)s = \nabla_X \nabla_Y s - \nabla_Y \nabla_X s - \nabla_{[X,Y]} s \]











\section{Characteristic Classes of Vector Bundles}


The transformation rules in \eqref{connection 1-form transformation} and \eqref{curvature 2-form transformation} tells us that all of the $\omega_\alpha$'s and $\Omega_\alpha$'s cannot necessarily be patched together to make global $\gl(\mathbb R^n)$-valued forms. If our transition functions happen to take values in an abelian Lie group, then we could make a global curvature form, but otherwise this is not possible.

However, there is an easy way to form a globally defined, real differential form out of the local expressions of the curvature. We say that a polynomial function $f : \gl(\mathbb R^n) \rightarrow \mathbb R$ of $n^2$ variables is invariant if $f(X) = f(A^{-1}XA)$ for all $X,A \in \gl(\mathbb R^n)$. The collection of such polynomials forms a commutative algebra, denoted by $I(n)$. Examples of such polynomials are $f(X) = \det X$ and $f(X) = \tr X$. 

Let $\sigma_1,\ldots,\sigma_n$ denote the elementary symmetric polynomials in $n$ variables. That is
\[ \sigma_0 = 1 \ \ \ \ \ \sigma_1 = x_1 + \cdots x_n \ \ \ \ \ \sigma_2 = x_1x_2 + x_1x_3 + \cdots + x_{k-1}x_n \ \ \ \cdots \ \ \ \sigma_n = x_1 \cdots x_n \]
We can extend the $\sigma_i$'s to maps $\sigma_i : \gl(\mathbb R^n) \rightarrow \mathbb R$ by defining $\sigma_i(X) = \sigma_i(\lambda_1,\ldots,\lambda_n)$, where $\lambda_1,\ldots,\lambda_n$ are the eigenvalues of $X$ (the order does not matter since $\sigma_i$ is symmetric). In particular, $\sigma_i(X)$ is simply the coefficient of $t^{k-i}$ in the characteristic polynomial of $\det(tI - X)$, so clearly $\sigma_i$ is an invariant polynomial on $\gl(\mathbb R^n)$. The elementary symmetric polynomials turn out to generate all of the invariant polynomials.
\begin{prop}
The algebra $I(n)$ is isomorphic to a polynomial ring
\[ I(n) \cong \mathbb R[\sigma_1,\ldots,\sigma_n] \]
\end{prop}
\begin{proof}(Sketch)
\end{proof}

There is another generating set of invariant polynomials of $I(n)$, which will be useful. For each $1 \leq i \leq k$ let $s_i$ be the invariant polynomial defined by
\[ s_i(X) = \tr X^i \]
We can easily express some of the $s_i$'s in terms of $\sigma_i$'s, for small $i$. For example, 
\[ s_1 = \sigma_1 \ \ \ \ \ s_2 = \sigma_1^2 - 2\sigma_2 \ \ \ \ \ s_3 = \sigma^3 - 3\sigma_1\sigma_2 + 3\sigma_3 \]
Formulas like these hold for all $s_i$'s, and conversely the $\sigma_i$'s can be written as polynomials of the $s_i$'s. This implies 
\begin{prop}
The algebra $I(n)$ is isomorphic to a polynomial ring
\[ I(n) \cong \mathbb R[s_1,\ldots,s_n] \]
\end{prop}






%%%%%%%%%%%%%%%%%%%%%%%
%%%%%%%%%%%%%%%%%%%%%%%
%%%%%%%%%%%%%%%%%%%%%%%
%%%%%%%%%%%%%%%%%%%%%%%

\todo{rewrite this trash}


Let $f \in I(n)$ be an invariant polynomial of degree $k$. Then $f(\Omega_\alpha)$ is a $2k$-form on $U_\alpha$. For two overlapping charts $U_\alpha,U_\beta$ we have that $f(\Omega_\alpha)$ and $f(\Omega_\beta)$ are related by
\[ f(U_\beta) = f(g_{\alpha\beta}^{-1} U_\alpha g_{\alpha\beta}) = f(U_\alpha) \]
Therefore the $f(\Omega_\alpha)$'s can be patched together to form a global form, which we denote by $f(\Omega) \in \Omega^{2k}(M)$.
\begin{prop}
For any invariant polynomial $f \in I(n)$, the form $f(\Omega)$ is closed.
\end{prop}
\begin{proof}
Since $I(n)$ is generated by the $s_i$'s it suffices to show $d s_i(\Omega)=0$. To do this we use Bianchi's identity:
\begin{align*}
	ds_i(\Omega) &= d \tr(\Omega^i) \\
	             &= \tr (d\Omega^i) \\
	             &= \tr( d\Omega \wedge \Omega^{i-1} + \Omega \wedge d\Omega \wedge \Omega^{i-2}+ \cdots + \Omega^{i-1} \wedge d\Omega ) \\
	             &= \tr( (\Omega \wedge \omega - \omega \wedge \Omega) \wedge \Omega^{i-1} + \Omega \wedge (\Omega \wedge \omega - \omega \wedge \Omega) \wedge \Omega^{i-2}+ \cdots + \Omega^{i-1} \wedge (\Omega \wedge \omega - \omega \wedge \Omega) ) \\
	             &= \tr(-\omega \wedge \Omega^i + \Omega^i \wedge \omega) \\
	             &= -\tr(\omega \wedge \Omega^i) + \tr(\omega \wedge \Omega^i) = 0
\end{align*}
\end{proof}







\section{Connections on Principal Bundles and Curvature}

Well will now attempt to repeat the program of connections and curvature we implemented in vector bundles for the more general case of fiber bundles. Unfortunately we cannot get very far on a general principal bundle, but we get a very rich theory on principal bundles. The theory is similar to what we got for vector bundles, but different enough to be interesting.

First we will consider something that may seem a little off topic, but will be important soon. Suppose a Lie group $G$ acts on a manifold $M$, and denote this action by $\tau : G \times M \rightarrow M$. Then the pushforward is a map $\tau_* : \mathfrak g \times TM \rightarrow TM$. For an element $X \in \mathfrak g$ of the Lie algebra of $G$, we will define a vector field $X^* \in \Gamma(TM)$. For $p \in M$ it is defined by
\[ X^*_p = \tau_{*(e,p)}(X,0) \]
%Here is another way of thinking about this induced vector field. A vector $X \in \mathfrak g$ determines a 1-parameter subgroup of diffeomorphisms $\phi_t : G \rightarrow G$. This induces a 1-parameter family of diffeomorphisms $\varphi_t : M \rightarrow M$ by $\varphi_t(p) = \phi_t(e) \cdot p$. Then we define
%\[ X_p^* = (\varphi_t)_{*p} \]
One can prove the following about the association of $X^*$ to $X$.
\begin{prop}
\label{properties of fundamental vector fields}
The association $X \mapsto X^*$ defines a Lie algebra homomorphism $\sigma : \mathfrak g \rightarrow \Gamma(TM)$. Further, if $G$ acts effectively, then $\sigma$ is an embedding, and if $G$ acts freely, then $X^*$ does not vanish if $X \neq 0$.
\end{prop}

Let us consider the case of a trivial fiber bundle, $F \rightarrow E = B \times F \stackrel{\pi}{\rightarrow} B$, where $\pi$ is projection onto $B$. Also let $\pi' : B \times F \rightarrow F$ be projection onto $F$. At a point $u = (b,p) \in E$ we see that the tangent space $T_u E$ splits as
\[ T_u E = \ker \pi'_{*u} \oplus \ker \pi_{*u} \cong T_b B \oplus T_p F \]
The first summand can be thought of as the vectors ``along'' the base, and the second summand can be thought of as vectors ``along'' the fiber. For this reason we define $H_u = \ker \pi'_{*u}$ and call these vectors horizontal, and we define $V_u = \ker \pi_{*u}$ and call these vertical. The tangent bundle $TE$ splits as $H \oplus V$, where $H$ and $V$ are the sub-bundles determined by $H_u$ and $V_u$, so every tangent vector of $E$ can be uniquely written as a sum of a horizontal and vertical vector.

Now consider a general fiber bundle $F \rightarrow E \stackrel{\pi}{\rightarrow} B$. At a point $u \in E$ we still have a natural choice of vertical vectors in $T_u E$, which we will denote by $V_u = \ker \pi_{*u}$. The $V_u$'s form a sub-bundle of $TE$, which we will denote by $V$. However, we no longer have a natural map $\pi' : E \rightarrow F$ with which we can define horizontal vectors. This leads us to define a connection on $E$ as a sub-bundle $H$ of $TE$ such that $TE = H \oplus V$. This means we are choosing a sub-bundle of $TE$ whose fibers are transverse to the fibers of the vertical sub-bundle. This type of connection is also sometimes called an Ehresmann connection.

\begin{prop}
Every fiber bundle admits a connection
\end{prop}
\begin{proof}
Give the total space $E$ a Riemannian metric. Then we can define $H$ to be the orthogonal complement of $V$.
\end{proof}

Since a principal bundle is just a fiber bundle, the above definition of connection clearly applies to principal bundles. However, principal bundles have extra structure, and so we want our connection to be compatible with this extra structure. First we recall the definition and properties of principal bundles.

A principal $G$-bundle is a fiber bundle $P \stackrel{\pi}{\rightarrow} M$ such that $G$ acts freely and transitively on the right on $P$, and the action preserves the fibers. In this case we clearly see that $P/G \cong M$, and each fiber is diffeomorphic to $G$. A local chart of the principal bundle is an open set $U \subseteq M$ and diffeomorphism $\varphi : \pi^{-1}(U) \rightarrow U \times G$ such that
\[ \varphi (ug) = \varphi(u)g \]
for all $g \in G$ and $u \in \pi^{-1}(U)$, where the action of $G$ on $U \times G$ is by right translation of the second factor. A condensed way of saying this is that $\varphi$ is a chart in the sense of fiber bundles that is also equivariant with respect to the action of $G$.

\begin{prop}
A principal $G$-bundle is trivial if and only if it admits a section.
\end{prop}
\begin{proof}
One direction is obvious. Suppose $P \rightarrow M$ is a principal $G$-bundle, and let $s : M \rightarrow P$ be a section. We define a bundle map $\tilde{f} : P \rightarrow M \times G$ in the following way. For $u \in P$, let $g$ be the unique element in $G$ such that $ug = s(\pi(u))g$. Existence of such an element is guaranteed by the transitivity of the action of $G$, and uniqueness is guaranteed by the freeness of the action. Then define $\tilde{f}(u) = (\pi(u),g)$. This map is clearly smooth, bijective, and a bundle map, and so it follows that it is an equivalence of bundles.
\end{proof}

A connection on a principal $G$-bundle $P \stackrel{\pi}{\rightarrow} M$ is a connection $H$ in the fiber bundle sense, such that each $H_u$ is invariant under the action of $G$. This last condition means that for each $u \in P$ and $g \in G$ we have $H_{ug} = (R_g)_* H_u$, where $R_g$ is right translation by $g$.

Like in the case of vector bundles, we can get a differential form representation of connections in a principal $G$-bundle. Let $P \rightarrow M$ be a principal $G$-bundle with a connection $H$. First we claim that the vertical subspaces $V_u$ can be naturally identified with $\mathfrak g$. Let $X \in \mathfrak g$ and $X^*$ the induced vector field on $P$. Since $G$ acts freely on $P$, by \cref{properties of fundamental vector fields} we have that $X_u^*$ is non-zero. Therefore the association $X \mapsto X_u^*$ is injective. We claim that the image of this association is contained in $V_u$. Since $G$ preserves fibers we have the following commutative diagram:
\[
\xymatrix
{
	P \times G \ar[r]^-{\tau} \ar[d]_{\proj} & P \ar[d]^{\pi} \\
	P \ar[r]_{\pi} & M 
}
\]
\unfinished
Since the dimensions of the two vector spaces $\mathfrak g$ and $V_u$ are obviously equal, we have a canonical isomorphism $\mathfrak g \rightarrow V_u$ given by $X \mapsto X_u^*$. 

We define a 1-form $\omega \in \Omega^1(P;\mathfrak g)$ in the following way. For $u \in P$ and $X_u \in T_u P$, let $\omega(X_u)$ be the composition
\[ T_u P \longrightarrow V_u \longrightarrow \mathfrak g \]
where the first map is the projection of $T_u P$ onto the vertical subspace (possible since we have a connection $H$), and the second map is the natural isomorphism. The 1-form $\omega$ is called the connection 1-form. It is clear then that $H = \ker \omega$. Conversely, given any form $\omega \in \Omega^1(P;\mathfrak g)$, what conditions does it have to satisfy so that $H = \ker \omega$ is a connection?

\begin{prop}
If $H$ is a connection on a principal bundle $P$, then a 1-form $\omega \in \Omega^1(P;\mathfrak g)$ is determined such that
\begin{enumerate}
	\item $\omega(X^*) = X$ for all $X \in \mathfrak g$
	\item $R_g^* \omega = \Ad(g^{-1}) \omega$
\end{enumerate}
Conversely, given any $\omega$ that satisfy these two conditions, then $H = \ker \omega$ is a connection on $P$.
\end{prop}

Given a trivial principal bundle $P = M \times G \stackrel{\pi}{\rightarrow} M$, what is the trivial connection? In terms of horizontal subspaces it is clearly the kernel of the projection $M \times G \stackrel{\pi_2}{\rightarrow} G$. In terms of the connection 1-form it is just the pullback of the Maurer-Cartan form $\omega_0$. So, let $\omega = \pi_2^* \omega_0$ be the trivial connection 1-form. Then, by \eqref{Maurer-Cartan equation for the Maurer-Cartan form} we have 
\[ d\omega = d \pi_2^* \omega_0 = \pi_2^* d\omega_0 = \pi_2^*\left( -\frac{1}{2} [\omega_0,\omega_0] \right) = -\frac{1}{2} [\pi_2^*\omega_0,\pi_2^*\omega_0] = -\frac{1}{2} [\omega,\omega] \]
For a general principal bundle and connection 1-form, the above will hold on each fiber, but will not necessarily hold globally. The failure for this to hold globally is measured by the curvature of the connection.

So, let $P \stackrel{\pi}{\rightarrow} M$ be a principal bundle with connection bundle $H$ and induced 1-form $\omega$. Let $h : TP \rightarrow H$ denote the horizontal projection of vectors on $P$ to $H$ along the vertical sub-bundle $V$. We will also sometimes use the notation $X_h$ to denote the horizontal part of a vector $X$ on $P$. Let $\Omega$ be the $\mathfrak g$-valued 2-form defined by
\[ \Omega := (d\omega) \circ h \times h \]
That is, we first project the vectors to the vertical bundle, and the evaluate $d\omega$ on them. The connection 1-form and curvature 2-form are related more explicitly by the \textbf{structure equation}:
\begin{prop}[Structure Equation]
If $\omega$ is the connection 1-form on a principal bundle and $\Omega$ is the curvature 2-form, then for all $X,Y \in T_u P$ we have
\[ d\omega(X,Y) = -\frac{1}{2} [\omega(X),\omega(Y)] + \Omega(X,Y) \]
\end{prop}
So, another way of defining the curvature 2-form is by the equation
\begin{equation}
\label{explicit principal bundle curvature}
d\omega = -\frac{1}{2} [\omega,\omega] + \Omega
\end{equation}
Take a basis $X_1,\ldots,X_n$ with structure constants $c_{ij}^k$. Suppose we write the connection 1-form as $\omega = \omega^i X_i$, where $\omega^i$ are real 1-forms, and similarly $\Omega = \Omega^i X_i$, where $\Omega^i$ are real 2-forms. Then the curvature can be written as
\begin{equation}
\label{principal bundle structure equation in basis}
d\omega^i = -\frac{1}{2} c_{jk}^i \omega^j \wedge \omega^k + \Omega^i
\end{equation}
The following properties of the curvature 2-form are easy to prove, two of which are immediate from the definition of curvature.
\begin{prop}
Let $\omega,\Omega$ be the connection and curvature forms associated to a principal bundle $P \rightarrow M$. Then
\begin{enumerate}
	\item For $g \in G$, $(R_g)^* \Omega = \Ad(g^{-1})\Omega$.
	\item For any vectors $X,Y \in T_u P$ we have $\Omega(X,Y) = \Omega(X_h,Y_h)$. 
	\item If $X,Y$ are horizontal vectors in $T_u P$, then $\Omega(X,Y) = -\frac{1}{2}[\omega(X),\omega(Y)]$.
	\item(Bianchi's Identity) $d\Omega = [\Omega,\omega]$.
\end{enumerate}
\end{prop}
\begin{proof}
\sloppyspace
\begin{enumerate}
	\item 
	\item Obvious from definition.
	\item If $X,Y$ are horizontal, then $\omega(X)=\omega(Y)=0$, so this follows from \eqref{explicit principal bundle curvature}.
	\item Differentiating both sides of \eqref{explicit principal bundle curvature} gives
	\begin{align*}
		d\Omega  &= d^2 \omega + \frac{1}{2} d[\omega,\omega] \\
					&= \frac{1}{2} \left( [d\omega,\omega] - [\omega,d\omega] \right) \\
					&= [d\omega,\omega] \\
					&= \left[ -\frac{1}{2} [\omega,\omega] + \Omega,\omega \right] \\
					&= [\Omega,\omega]
	\end{align*}
\end{enumerate}
\end{proof}







\section{Characteristic Classes of Principal Bundles}

Unfortunately there is not any workable theory of characteristic classes in general fiber bundles. However, principal $G$-bundles provide a rich theory. We will review some basic Chern-Weil theory of characteristic classes for principal $G$-bundles.



Let us continue using the basis $X_1,\ldots,X_n$ from before, and now let $\theta^1,\ldots,\theta^n$ be its dual basis. Let $\omega$ be a connection 1-form with curvature 2-form $\Omega$, and express $\omega = \omega^i X_i$ and $\Omega = \Omega^i X_i$, where $\omega^i,\Omega^i$ are real differential forms. Suppose we are given a linear map $\alpha : \mathfrak g \rightarrow \mathbb \mathbb R$, i.e. an element of $\mathfrak g^*$. Then for each $u \in P$, the composite $\alpha \circ \omega : T_u P \rightarrow \mathbb R$ depends smoothly on $u$, hence $\alpha \circ \omega$ determines a 1-form, which we shall denote by $\overline{\omega}(\alpha) \in \Omega^1(P)$. We clearly have $\overline{\omega}(\theta^i) = \omega^i$. This association depends linearly on $\alpha$, hence the connection 1-form induces a linear map, which we will denote by $\overline{\omega} : \mathfrak g^* \rightarrow \Omega^1(P)$. We can extend this to exterior powers to give a linear map 
\[ \overline{\omega} : \Lambda^* \mathfrak g \rightarrow \Omega^*(P) \]
In particular, we have
\[ \overline{\omega}(\theta^{i_1} \wedge \ldots \wedge \theta^{i_k}) = \omega^{i_1} \wedge \cdots \wedge \omega^{i_k} \]
In general, this induced map does not commute with the exterior derivative, i.e. $d \circ \overline{\omega}$ does not necessarily equal $\overline{\omega} \circ d$, where this last $d$ is defined on $\mathfrak g^* = \Omega^1(G)$. The failure of this to commute is related to the curvature.

So, lets try to repeat these constructions with the curvature 2-form $\Omega$. For $\alpha \in \mathfrak g^*$ we see that $\alpha \circ \Omega : T_u P \times T_u P \rightarrow \mathbb R$ depends smoothly on $u$, hence defines a 2-form, denoted by $\Omega(\alpha) \in \Omega^2(P)$. This gives an induced map, which we shall denote by $\overline{\Omega} : \mathfrak g^* \rightarrow \Omega^2(P)$. In particular, we see that $\overline{\Omega}(\theta^i) = \Omega^i$. Using \eqref{Maurer-Cartan equations in coordinates} and \eqref{principal bundle structure equation in basis} we get
\begin{align*}
	\Omega^i &= d\omega^i + \frac{1}{2} c_{jk}^i \omega^j \wedge \omega^k \\
	         &= d\omega^i + \frac{1}{2} c_{jk}^i \overline{\omega}(\theta^j \wedge \theta^k) \\
	         &= d\omega^i + \overline{\omega}\left( \frac{1}{2} c_{jk}^i \theta^j \wedge \theta^k \right) \\
	         &= d\omega^i - \overline{\omega}(d\theta^i)
\end{align*}
This implies that for any $\alpha \in \mathfrak g^*$ we have
\[ \overline{\Omega}(\alpha) = (d \circ \overline{\omega} - \overline{\omega} \circ d)(\alpha) \]
This shows how curvature can be thought of as measuring the commutativity of $\overline{\omega}$ with the exterior derivative $d$. 





\unfinished









\section{Spin Geometry and the Dirac Operator}


The orthogonal group $O(k+1)$ acts on $S^k$ in the obvious way, and the stabilizer of any point $p \in S^n$ is the set of rotations and reflections through the plane perpendicular to $p$, which is isomorphic to $O(k)$. Therefore we have $O(k+1)/O(k) = S^k$, and so we have a fibration $O(k) \rightarrow O(k+1) \rightarrow S^k$. This gives a long exact sequence of homotopy groups
\[ \cdots \longrightarrow \pi_2(S^k) \longrightarrow \pi_1(SO(k)) \longrightarrow \pi_1(SO(k+1)) \longrightarrow \pi_1(S^k) \]
If $k \geq 3$, then $\pi_2(S^k) = \pi_1(S^k) = 0$, hence $\pi_1(SO(k)) = \pi_1(SO(k+1))$. We know that $SO(3) = \mathbb RP^3$, and that $\pi_1(\mathbb RP^3) = \mathbb Z/2$, hence $\pi_1(SO(k)) = \mathbb Z/2$ for $k \geq 3$. This implies that $SO(k)$ has a two-fold universal Lie group cover; it is called the \textbf{Spin group} and we denote it by $\rho : \Spin(k) \rightarrow SO(k)$. 

Let $E \stackrel{\pi}{\rightarrow} M$ be an orientable rank $k$ vector bundle with transition functions $\lcb g_{\alpha\beta} \rcb$, which we assume take values in $SO(k)$ by orientability. We can always choose our cover $\lcb U_\alpha \rcb_{\alpha \in \Lambda}$ of the base manifold so that any finite intersection of open sets from this collection is contractible (such a cover is called \textbf{good}), and so we assume that our cover satisfies this. Then by the lifting theorem in covering space theory we can lift each transition function to a map $\tilde{g}_{\alpha\beta} : U_\alpha \cap U_\beta \rightarrow \Spin(k)$ such that $\rho \circ \tilde{g}_{\alpha\beta} = g_{\alpha\beta}$, where $\rho$ is the covering map. Unfortunately the collection of maps $\lcb \tilde{g}_{\alpha\beta} \rcb$ will not always satisfy the cocycle condition \ref{cocycle condition}. But, if they do, we can use them as transition functions for a principal $\Spin(k)$-bundle. Thus we define a \textbf{spin structure} on $E$ to be the principal $Spin(k)$-bundle formed from lifting the transition functions of $E \rightarrow M$ to $Spin(k)$ such that they satisfy the cocycle condition, and denote it by $E_{Spin(k)}$. A general vector bundle can admit many spin structures or no spin structure. An orientable manifold $M$ is said to be \textbf{spin} if the tangent bundle $TM \rightarrow M$ has a spin structure. 

For any triple of indices $(\alpha,\beta,\gamma) \in \Lambda^3$ we define the maps $\eta_{\alpha\beta\gamma} = \tilde{g}_{\alpha\beta} \tilde{g}_{\beta\gamma} \tilde{g}_{\gamma\alpha}$. Since $\rho \circ \eta_{\alpha\beta\gamma} = g_{\alpha\beta} g_{\beta\gamma} g_{\gamma\alpha} = I$ we see that $\eta_{\alpha\beta\gamma}$'s image is contained in $\rho$'s kernel, hence
\[ \eta_{\alpha\beta\gamma} : U_\alpha \cap U_\beta \cap U_\gamma \rightarrow \ker \rho \cong \mathbb Z/2 \]
One can check that the collection of these maps $\lcb \eta_{\alpha\beta\gamma} \rcb$ satisfy the relations to be a \Cech cocycle in $H^2(M,\mathbb Z/2)$, and one can show that this cohomology element is precisely the second Steifel-Whitney class of the bundle $w_2(E)$. If this class is trivial, then all the maps are constant mapping into the identity, which of course implies the coycle condition is satisfied for the collection $\lcb \tilde{g}_{\alpha\beta} \rcb$. Therefore a spin structure exists on a vector bundle if and only if its second Steifel-Whitney class is trivial.

Let $E_{SO(k)}$ be the frame bundle of $E$, and suppose $E$ is equipped with a spin structure $E_{Spin(k)}$. We will show that there is a bundle map $f : E_{Spin(k)} \rightarrow E_{SO(k)}$ such that restricted to a fiber is a 2:1 covering map. For a point $p \in M$ take charts $\varphi_\alpha : \pi^{-1}(U_\alpha) \rightarrow U_\alpha \times SO(k)$ and $\tilde{\varphi}_\alpha : \tilde{\pi}^{-1}(U_\alpha) \rightarrow U_\alpha \times Spin(k)$, and for a point $x_p$ in the fiber of $E_{Spin(k)}$ over $p$ define
\[ f(x_p) = \varphi_{\alpha}^{-1} \circ (\id \times \rho) \circ \tilde{\varphi}_{\alpha}(x_p) \]
We need to check that this is well-defined. Suppose we had another pair of charts $\varphi_\beta$ and $\tilde{\varphi}_\beta$, and suppose $(p,Y) \in U_\beta \times Spin(k)$ such that $\tilde{\varphi}_\beta^{-1}(p,Y) = x_p$. Note that $\varphi_\alpha \circ \varphi_\beta^{-1}(p,X) = (p,g_{\alpha\beta}(p) \cdot X)$, and similarly for $\tilde{\varphi}_\alpha \circ \tilde{\varphi}_\beta^{-1}$. Therefore $f$ being well-defined is equivalent to
\begin{align*}
	 & \varphi_\alpha^{-1} \circ (\id \times \rho) \circ \tilde{\varphi}_\alpha(x_p) = \varphi_\beta^{-1} \circ (\id \times \rho) \circ \tilde{\varphi}_\beta(x_p) \\
	\Longleftrightarrow \ \ \ \ \ & \varphi_\beta \circ \varphi_\alpha^{-1} \circ (\id \times \rho) \circ \tilde{\varphi}_\alpha \circ \tilde{\varphi}_\beta^{-1}(p,Y) = (\id \times \rho)(p,Y) \\
	\Longleftrightarrow \ \ \ \ \ & (\id \times g_{\beta\alpha} \cdot) \circ (\id \times \rho) (p, \tilde{g}_{\alpha\beta}(p) \cdot Y) = (p,\rho(Y)) \\
	\Longleftrightarrow \ \ \ \ \ & (\id \times g_{\beta\alpha} \cdot) (p, g_{\alpha\beta}(p) \cdot \rho(Y)) = (p,\rho(Y)) \\
	\Longleftrightarrow \ \ \ \ \ & (p, g_{\beta\alpha}(p) \cdot g_{\alpha\beta}(p)(\rho(Y))) = (p,\rho(Y)) \\
	\Longleftrightarrow \ \ \ \ \ & (p,\rho(Y)) = (p,\rho(Y))
\end{align*}
Therefore $f$ is well-defined. Conversely, a principal $Spin(k)$-bundle $P \rightarrow M$ that admits a morphism $f : P \rightarrow E_{SO(k)}$ such that $f$ restricts to a 2:1 covering map on fibers determines a spin structure on $E \rightarrow M$. Yet another description of spin structures, in view of our discussion on associated bundles, is a principal $Spin(k)$-bundle $P \rightarrow M$ such that $P \times_\rho SO(k)$ is isomorphic to $E_{SO(k)}$ as principal $SO(k)$-bundles.

There is yet another way of looking at spin structures, which is more universal and gives an easy proof that if a spin structure exists, then the 2nd Steifel-Whitney class vanishes. Let $B\rho$ denote the map induced on the classifying spaces $BSpin(k) \rightarrow BSO(k)$, where $\rho$ is the double cover of $SO(k)$. If $E \rightarrow M$ is an orientable, rank $k$ vector bundle, then there is a classifying map $f : M \rightarrow BSO(k)$ such that $E \cong f^* ESO(k)$. A spin structure on $E$ is simply a lift $\tilde f : M \rightarrow BSpin(k)$ such that $B\rho \circ \tilde f = f$. One can show that the $\mathbb Z/2$ cohomology of $BSO(k)$ is a polynomial algebra $\mathbb Z/2[w_2,\ldots,w_k]$, where $w_i \in H^i(BSO(k);\mathbb Z/2)$ are the universal Steifel-Whitney classes. Since $Spin(k)$ is simply connected, and $\pi_n(G)=\pi_{n+1}(BG)$ for any topological group $G$ and its classifying space $BG$, we see that $\pi_2(BSpin(k))=0$, hence $H_2(BSpin(k))=0$, and so $H^2(BSpin(k),\mathbb Z/2) = 0$. Then the 2nd Steifel-Whitney class of $E$ is $f^*w_2 = B\rho^* \circ \tilde f^*w_2 = 0$.

Suppose $E \rightarrow M$ has a spin structure $E_{Spin(k)}$, and let $\rho : \Spin(k) \rightarrow GL(V)$ be any representation of $Spin(k)$ on $V$. The vector bundle $P_{Spin(k)} \times_\rho V \rightarrow M$ is called the bundle of \textbf{spinors of type $\rho$}, and sections of this bundle are called \textbf{spinor fields of type $\rho$}.













\section{Mathematics-to-Physics Gauge Theory Dictionary}






\section{Maxwell's Electricity and Magentism as a Gauge Theory}








\section{Chern-Simons Theory}




\section{Seiberg-Witten Equations}
















\end{document}






